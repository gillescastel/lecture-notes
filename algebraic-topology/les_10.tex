\lesson{10}{di 17 dec 2019 10:26}{Ham sandwich theorem}

\begin{theorem}
    Let $f: S^1 \to S^{1}$ such that $f$ is antipode preserving: $f(A x) = Af(x)$.
    Then $d(f)$ is odd.
\end{theorem}
\begin{proof}
    Picture:
\begin{figure}[H]
    \centering
    \incfig{antipode-preserving-maps}
    \label{fig:antipode-preserving-maps}
\end{figure}
What is $f_*: H_1(S^{1}) \to  H_1(S^{1})$?
$H_1(S^{1}) \cong \Z = \left<\alpha \right> = \left< c + d \right>$.
Note that $d = A  \circ c$.

Now, Let's look at the image of $c$ under  $f$.
The only thing we know, is that it starts at $f(a)$ and ends at $f(b) = A f(a)$.
Let  $\gamma$ be the path that is the half circle, counterclockwise from $f(b)$ to $f(a)$.
Then $f  \circ  c + \gamma \in Z_1(S^{1})$.
Indeed, $\partial(f  \circ  c + \gamma) = f(b) - f(a) + f(a) - f(b) = 0$.
Then $\left<f  \circ  c + \gamma \right> \in H_1(S^{1})$.
This has to be equal to $m \left<\alpha \right>$ for some $m \in \Z$.

Now, $A_*: H_1(S^{1}) \to  H_1(S^{1})$.
We know that $d(A_*) = (-1)^{2} = 1$, so this is just the identity map:
\[
    A_*(\left<\alpha \right>) = A_* \left<c + d \right> = \left<A  \circ  c + A  \circ  d \right> = \left<d + c \right> = \left<\alpha \right>
.\] 
So \[
m \left<\alpha \right> = A_* \left<f  \circ  c + \gamma \right>
= \left<A  \circ  f  \circ  c + A  \circ  \gamma \right>
.\]
Now, $A  \circ  \gamma$ is the path from $f(a)$ to $f(b)$, going counter clockwise. Now, we assume $A  \circ  f = f  \circ  A$, so
\[
m \left<\alpha \right> = \left<f  \circ  d + A  \circ  \gamma \right>
.\] 
Now, also
\begin{align*}
    2 m \left<a \right> &= \left<f  \circ  c + \gamma \right> + \left< f  \circ  d + A  \circ  \gamma \right>\\
                        &= \left<f  \circ  c + f  \circ  d + \gamma + A  \circ  \gamma \right>\\
                        \intertext{$\gamma + A  \circ  \gamma$ is a one cycle, and $f  \circ  c + f  \circ  d$ is also a one cycle}
                        &= \left<f  \circ  c + f  \circ  d \right> + \left<\gamma + A  \circ  \gamma \right>\\
                        &= f_* \left< c + d \right> + \left< \alpha \right>\\
                        &= f_* \left< \alpha \right> + \left<\alpha \right>
.\end{align*} 
So
\[
    f_* \left<\alpha \right> = (2m - 1) \left<\alpha \right>
,\] 
for some integer $m$.
\end{proof}
\marginpar{TODO: In problem: say that one cycles}

\begin{remark}
    Exercise: even maps have even degrees: $f(A x) = f(x)$.
\end{remark}
\begin{remark}
    We don't use $- x$ for $Ax$ because it could be misleading: this $-$ has nothing to do with the $-$ from the abelian homology group.
\end{remark}

\begin{theorem}
    Let $f: S^{n} \to  S^{n}$ such that $f(Ax) = A f(x)$.
    Then  $d(f)$ is odd.
\end{theorem}
\begin{proof}
    We don't prove this.
\end{proof}

\begin{theorem}
    There is no continuous antipodal preserving map from $S^2 \to  S^{1}$.
\end{theorem}
\begin{proof}
    Suppose there is such a map.
    \begin{figure}[H]
        \centering
        \incfig{no-antipodal-preserving-map-from-s2-to-s1}
        % \caption{no antipodal preserving map from s2 to s1}
        \label{fig:no-antipodal-preserving-map-from-s2-to-s1}
    \end{figure}
\[
    \begin{tikzcd}
        S^2 \arrow[r, "f"] & S^{1}\\
        S^{1} \arrow[u, "i"] \arrow[ur, swap, "f|_{S^{1}}"] &
    \end{tikzcd}
\]
Then we have
\[
    \begin{tikzcd}
        H_1(S^{2}) = 0 \arrow[r, "f_*"] &H_1(S^{1}) = \Z\\
        H_1(S^{1}) = \Z \arrow[u, "i_*"] \arrow[ur, swap, "\cdot (2k+1)"]
    \end{tikzcd}
\]
But $(f_*  \circ  i_*)(1) = 0$, but $(f|_{S^{1}})_*(1) = 2k+1$, which is odd, and certainly not $0$.
This works for all higher dimensional spheres to $S^1$.
\end{proof}


\begin{theorem}[Borsuk-Ulam theorem]
    \emph{There are two antipodal points on the earth such that both the temperature and air pressure are the same.}
    If $f: S^{2} \to  \R^2$ is continuous.
    Then there exists $x \in S^2$ such that $f(x) = f(-x)$.
\end{theorem}
\begin{proof}
    Suppose $f(x) \neq f(-x)$ for all  $x \in S^2$.
    Then  we define
    \[
        g: S^2 \to  \R^2: x \mapsto \frac{f(x) - f(-x)}{\|f(x) - f(-x)\|} \in S^1
    .\] 
    Then $g(-x) = -g(x)$, which is a contradiction with the previous theorem.
\end{proof}
\begin{remark}
    Also holds for $S^{n}\to \R^n$.
\end{remark}

\begin{theorem}[Ham Sandwich theorem]
    \emph{Suppose you give me two pieces of break and 1 slice of ham. Then it is possible to divide both the pieces of bread and the slice of ham in equal pieces by 1 straight cut of knife.}
\end{theorem}
\begin{proof}
    Consider for each $v \in S^{2}$ a plane $P_v \subset \R^3$.
    $P_v \perp v$ and  $P_v$ cuts the slice of ham exactly in two.
    We define the ``upper side'' of the plane to be the half to which $v$ is pointing to.

    If you have some weird ham which which you can cut in multiple places in half, then you take the middel of the line segment. This makes it unique.

    Note that $P_v = P_{-v}$.
\begin{figure}[H]
    \centering
    \incfig{ham-sandwich-theorem}
    % \caption{ham sandwich theorem}
    \label{fig:ham-sandwich-theorem}
\end{figure}

Now, consider 
\[
    f: S^{2} \to  \R^2: v \mapsto (f_1(v), f_2(v))
.\] 
Then $f_1(v)$ is the volume of bread $b_1$ above $P_v$.
Then $f_2(v)$ is the volume of bread $b_2$ above $P_v$.

Now, you should believe that $f_1$ and $f_2$ are continuous. (Proving this precisely needs measure theory etc.)
So, now, we can use the Borsak Ulam theorem.
So there exists a $v \in S^{2}$ such that $f(v) = f(-v)$.
So  $f_1(v) = f_1(-v)$, so volume of bread $b_1$ above $P_v$ is the volume of bread $b_1$ below $P_v$, and similar for  $f_2$.
This proves the Ham sandwich theorem.
\end{proof}

Last result.
\begin{theorem}
    Let $A_1, A_2, A_3$ be three closed subsets of $S^2$ such that $S^2 = A_1 \cup A_2 \cup A_3$.
    Then at least one of the $A_i$ contains a pair of antipodal points.
\end{theorem}
\begin{proof}
    Let $f: S^2 \to \R^2$ mapping $ x\mapsto (d(x, A_1), d(x, A_2))$.
    \[
        d(x, A_i) = \min_{y \in A_i} d(x, y)
    .\] 
    Then, there exists a point $x \in S^2$ such that $f(x) = f(-x)$.
    So
     \[
         d(x, A_1) = d(-x, A_1) \qquad d(x, A_2) = d(-x, A_2)
    .\] 
    \begin{description}
        \item[Possibility 1] $d(x, A_1) = 0$. Then $x \in A_1$ and $-x \in A_1$.
        \item[Possibility 2] $d(x, A_2) = 0$. Then $x \in A_2$ and $-x \in A_2$.
        \item[Possibility 3] $d(x, A_2) \neq 0 \neq d(x, A_1)$: $x, -x \not\in A_1$ and $x, -x \not\in A_2$. As $x \in S^2$, $x, -x \in A_3$.
    \end{description}
\end{proof}

\begin{remark}
    This also works for $2$ (Just take $A_3 = \O$), but this doesn't work for $4$.
\end{remark}
\begin{figure}[H]
    \centering
    \incfig{example-with-four-elements}
    \caption{Example with four elements}
    \label{fig:example-with-four-elements}
\end{figure}
\begin{remark}
    Time on schedule is when you have to enter the room.
    Hand over problems after exam.
\end{remark}
