%\lesson{8}{di 19 nov 2019 10:29}{}
 
%Recap: let $f \in C^{\infty}(M)$ and $p \in M$, then $ (df)_p := (f_*)_p \in T_p^{*} M $. 

Let $k\ge 0$. For every $p\in M$, consider the vector space $\bigwedge ^{k} T_p^{*} M$.

\begin{definition}[Differential form]
    A $k$-form on $M$ is a map
    \[
    \alpha: M \to  \bigsqcup_{p \in  M} \bigwedge ^{k} T_p^{*} M
    \] 
    such that
    \begin{itemize}
        \item $\alpha(p) \in  \Lambda^{k} T_p^{*}M$ for each $p$
        \item For any chart $(U,\phi)$, writing
        $$\alpha(p)=
        \sum_{1 \le   i_1 <  \ldots <  i_k \le  m}a_{i_1   \ldots i_k}
        dx_{i_1} \wedge \cdots \wedge dx_{i_k}|_p,$$        
          the coefficients $a_{i_1   \ldots i_k} \colon U\to \R$ are smooth.
    \end{itemize}
\end{definition}
\begin{remark}
    In other words: a $k$-form is a section of the vector bundle  $\bigwedge^{k} T^{*}M$.
\end{remark}
We denote $\Omega^{k}(M) = \{\text{$k$-forms on $M$}\}$. Notice that $\Omega^0(M)=C^{\infty}(M)$.

\begin{remark}
    Given $\alpha \in \Omega^{k}(M)$ and vector fields $X_1, \ldots, X_k \in 
    \mathfrak{X}(M)$, we define $\alpha(X_1, \ldots, X_k)\in C^{\infty}(M)$ by
    \[
        (\alpha(X_1, \ldots, X_k))(p) := \alpha(p)(X_1(p), \ldots, X_k(p))
    .\] 
    Notice that for all $f \in C^{\infty}(M)$,
    $\alpha( fX_1, \ldots X_k) = f \alpha(X_1, \ldots, X_k)$.
\end{remark}
\begin{definition}
    Let $F: M \to  N$ be a smooth map.
    The pullback of $k$-forms is $F^{*}: \Omega^{k}(N) \to \Omega^{k}(M)$ :
    \[
        (F^{*} \omega)(p)(v_1, \ldots v_k) = \omega(F(p))(
        (F_*)_p v_1, 
        \ldots,
        (F_*)_p v_k)
\]
where $p\in M$, $v_i\in T_pM$. 
\end{definition}
%\begin{eg}
%    On $\R$, $dx_1 \in \Omega^{1}(\R^2)$ looks like $(1\  0)$
%\end{eg}

We now study the pullback of differential forms on $\R^n$.
\begin{prop}
    Let $U \subset \R^m$ open, $V \subset \R^n$ open, $G : U \to  V$ smooth.
    Denote with $x_j$ the standard coordinates on $\R^m$ and by $y_i$ those on $\R^n$.
    Then
    \begin{itemize}
        \item[1)] $G^{*}(d y_i) = \sum_{j=1}^{m} \cfrac{\partial G_i}{\partial x_j} d x_j \in \Omega^{1}(U)$
        \item[2)] If $m =n$, then
            \[G^{*} (f(y) dy_1 \wedge \ldots\wedge dy_m) = (f  \circ  G) \det (\operatorname{Jac} G) dx_1 \wedge \ldots \wedge dx_m.\]
    \end{itemize}
\end{prop}
Above $\operatorname{Jac} G$ denotes the Jacobian of $G$, i.e. the matrix representing the derivative $DG$ of $G$.
\begin{proof}
    \begin{itemize}
        \item $G^{*}(dy_i)\left(\frac{\partial }{\partial x_j}\right) = dy_i \left( G_* \frac{\partial }{\partial x_j}\right) = dy_{i}(\sum_k \frac{\partial G_k}{\partial x_{j}} \frac{\partial }{\partial y_{k}}) = \frac{\partial G_i}{\partial x_{j}} $
        \item 
            $G^{*}( f d y_1 \wedge \ldots \wedge dy_m) = (G^{*} f) (G^{*} dy_1 \wedge \ldots \wedge G^{*} dy_m)$.
            Evaluating on $\frac{\partial }{\partial x_1},\dots,\frac{\partial }{\partial x_m}$, we get
            \[
                (G^{*} f) \det \underbrace{\left(G^{*}(dy_{i}) \left(\frac{\partial }{\partial x_j}\right)\right)}_{=\dfrac{\partial G_i}{\partial x_{j}} \text{ by above}}
                = (f  \circ  G) \det (\operatorname{Jac} G)
            .\] 

    \end{itemize}
\end{proof}
\section{Orientation and volume forms}
\begin{definition}[Oriented atlas]
    An oriented atlas is a smooth atlas $(U_{\alpha},\phi_{\alpha})$ s.t.\ $\det D(\phi_\beta  \circ  \phi_\alpha^{-1}) > 0$ for all $\alpha,\beta$.
\end{definition}
\begin{definition}[Orientation]
    An orientation is a choice of maximal oriented atlas.
\end{definition}
\begin{remark}
    Not all manifolds are orientable, e.g.\ the Möbius band and $\R\mathbb{P}^2$ are not.
\end{remark}
\begin{remark}
Let $V$ be vector space.
On $\{\text{ordered bases of $V$}\}$, there is an equivalence relation $(v_1, \ldots, v_m) \sim (w_1, \ldots, w_m)$ if change of basis has $\det > 0$.
An orientation of  $V$ is by definition a choice of one of the two equivalence classes. 
An ordered basis of an oriented vector space is positive if it belongs to the equivalence class that gives the orientation.


Now let $M$ be a manifold. An orientation on $M$ induces and orientation on each tangent space $T_pM$.
\end{remark}
\begin{definition}[Volume form]
    A volume form on $M^{m}$ is an $m$-form $\Omega$ such that  $\Omega (p) \neq 0$ for all  $p \in  M$.
\end{definition}
\begin{remark}
    If $\Omega$ is a volume form, then an other volume form looks like $f \Omega$, where $f : M \to  \R\setminus \{0\}$.
\end{remark}

\begin{prop}
    $M$ is orientable iff there exists a volume form $\Omega$
\end{prop}
\begin{proof}
    We denote by $\text{Vol}:=dx_1 \wedge \ldots \wedge dx_m$ the standard volume form on~$\R^m$.

    $\impliedby$: choose an atlas consisting of charts such that $(\phi^{-1}_\alpha)^{*}\Omega = f_\alpha \text{Vol}$, where $f_\alpha $ is a positive function.
    Then $\det D(\phi_\beta  \circ  \phi_\alpha^{-1}) > 0$ because of item 2 of the last proposition.
    
      $\implies$: given an oriented atlas  $\{(U_{j},\phi_{j})\}$, choose a partition of unity $\{e_{\alpha}\}$ subordinate to it, and define 
      $\Omega = \sum_{\alpha} e_\alpha \cdot(\phi_\alpha)^* \text{Vol}.$
 Then $\Omega$ is a volume form, because on $U_\alpha\cap U_\beta$ we have
 $$ (\phi_\beta)^* \text{Vol}    =(\phi_\alpha)^*\underbrace{(\phi_\beta\circ \phi_\alpha^{-1})^* \text{Vol}}_{\det (D(\phi_\beta\circ \phi_\alpha^{-1}))\cdot \text{Vol}}=(\text{a positive function})\cdot (\phi_\alpha^{-1})^*\text{Vol}.
 $$
\end{proof}
\section{Integration on manifolds}
 

Let $M^{m}$ be a oriented manifold.
We want to define $\int_M \omega$, where $\omega \in  \Omega^{m}(M)$ has compact support.

\paragraph{Step 1.}
Assume  the support  $\supp(\omega):=\overline{\{p\in M: \omega(p)\neq 0\}}$ is contained in one chart $(U,\phi)$ of the oriented atlas..
Write $(\phi^{-1})^{*} \omega = f dx_1 \wedge \ldots \wedge dx_m$ for some $f \in C^{\infty}(\phi(U))$.
\begin{definition}[Integral of top form supported in a chart]
    \[
        \int_M \omega := \int_{\phi(U)} (\phi^{-1})^{*} \omega := \int_{\phi(U)} f(x) \: dx_1 dx_2 \cdots dx_m,\]
where on the right hand side is the multiple Riemann integral of the function $f$ over $\phi(U)$.         
\end{definition}

Recall the transformation rule in $\R^{m}$.
Let $U, V \subset \R^m$ and $\theta: V \to  U$ a diffeomorphism.
Let $f \in C^{\infty}(U)$ be integrable.
Then
\[
    \int_U f(x) \: dx_1 \cdots dx_m = \int_V (f  \circ  \theta)(y) | \det (\operatorname{Jac} \theta)| \: dy_1 \cdots dy_m
.\] 

 
\begin{lemma}
    $\int_M \omega$ is independent of the choice of chart in the oriented atlas.
\end{lemma}
\begin{proof}
  Let $(V,\psi)$ be another chart as above, and denote $\theta:=\phi \circ \psi^{-1}$.
  \begin{figure}[H]
    \centering
    \incfig{tangent-vector-definition-chart-independent-ForDiffForms}
  %  \caption{Definition of a vector bundle}
    \label{tangent-vector-definition-chart-independent-ForDiffForms}
\end{figure}
We want to find $g\in C^{\infty}(\psi(V))$ such that $(\psi^{-1})^*\omega=
  g(y) \: dy_1 dy_2 \cdots dy_m$. To do so we compute
$$
  (\psi^{-1})^*\omega=(\phi^{-1}\circ \theta)^*\omega= \theta^*((\phi^{-1})^*\omega)=
 \underbrace{(f  \circ  \theta) \det (\operatorname{Jac} \theta)}_{\text{ so this is $g$}}  dy_1 \wedge \ldots \wedge dy_m
$$
using a Proposition from \S\ref{sec:diffformman} in the last equality.
By the transformation rule $$\int_{\phi(U)} f(x) \: dx_1 dx_2 \cdots dx_m=
 \int_{\psi(V)} \underbrace{(f  \circ  \theta)(y) | \det (\operatorname{Jac} \theta)|}_{=g} \: dy_1 \cdots dy_m,$$
 finishing the proof.
(The function on the right hand side equals $g$ since 
the absolute values  can be removed, due to the fact that $\psi$ and $\phi$ lie in the oriented atlas of $M$.)
\end{proof}


\paragraph{Step 2.}
For any $\omega \in \Omega^{m}(M)$ with compact support, let $\{(U_\alpha, \phi_\alpha)\} $ be an oriented atlas such that $\supp(\omega) \cap U_\beta \neq \emptyset$ for finitely many $\beta$.
Let  $e_\beta$ be a partition of unity subordinate to this cover.
Notice that  $\omega = (\sum e_\beta) \omega = \sum (e_\beta \omega)$.
 \begin{definition}[Integral of top form]
    \[
    \int_M \omega = \sum_\beta \int e_\beta \omega
    .\] 
\end{definition}
Notice that each summand  $\int e_\beta \omega$ was defined in Step 1, since $\supp(e_\beta \omega)\subset U_{\beta}$.

\begin{remark}
 One can show: this definition is independent of the choice of oriented atlas and partition of unity.
\end{remark}
