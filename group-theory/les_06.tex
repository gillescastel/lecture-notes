\lesson{6}{di 13 okt 2020 14:52}{}

Previous example for $ D_3$:
\begin{center}
\begin{tabular}{cccc}
    & $1$ &  $a$ &  $b$ \\ \hline
    $\chi_{1}$ & 1& 1 & 1\\
    $\chi_{2}$ &$1$ & $1$& $-1$\\
    $\chi_{3}$ & 2& $-1$& 0\\
\end{tabular}
\end{center}

Now we want to find the irreducible components of $\rho_\text{reg} : D_3 \to  \GL(\C[D_3])$.
We know that

\begin{align*}
    \chi_{\rho_\text{reg} }(e) &= |D_3|= 6\\
    \chi_{\rho_\text{reg} }(g) &= 0 \text{ for $g \neq e$ }
.\end{align*}

Remember that $(\chi_{\rho} = \chi_{i})_G$ counts the number of components equivalent to $\chi_{i}$.

This is easy because we have lot's of zeros:
\begin{align*}
    (\chi_{\rho_\text{reg}} | \chi_{1}) &= \frac{1}{6} ( 6 \cdot  1 + 0 + \cdots + 0) = 1\\
    (\chi_{\rho_\text{reg}} | \chi_{2}) &= \frac{1}{6} ( 6 \cdot  1 + 0 + \cdots + 0) = 1\\
    (\chi_{\rho_\text{reg}} | \chi_{3}) &= \frac{1}{6} ( 6 \cdot  2 + 0 + \cdots + 0) = 2
.\end{align*} 
So
\[
\chi_{\rho_\text{reg} } = \chi_{1} + \chi_{2} + 2 \chi_{3}
.\] 

\begin{theorem}
    Let $G$ be a finite group and $ \chi_{1}, \ldots, \chi_{r}$ be the irreducible characters of $G.$
    Then 
    \[
    \chi_{\rho_\text{reg} } = n_1 + \chi_{1} + \cdots + n_r \chi_{r}
    ,\] 
    with $n_1 = \chi_{i}(e) = $ dimension of the $i$th irreducible representation.
\end{theorem}
\begin{proof}
    See the calculation above.
    \begin{align*}
        (\chi_{\rho_\text{reg} } | \chi_{i}) &= \frac{1}{|G|} \sum \chi_{\rho_\text{reg} } (g) \overline{\chi_{i} (g)}\\
        &= \frac{1}{|G|} \chi_{\rho_\text{reg} }(e) \overline{\chi_i (e)}\\
        &=\frac{1}{|G|} |G| \overline{n_i} = n_i
    .\end{align*} 
\end{proof}
\begin{corollary}
    $|G| = \dim \rho_\text{reg} =  n_1^2 + n_2^2 + \cdots + n_r^2$
\end{corollary}
\begin{prop}
    $n_i  \mid  G$
\end{prop}
\begin{proof}
    Not in this course. But you may assume this during exercises.
\end{proof}

\nsection{3}{5}{Examples of character tables}

\paragraph{Abelian groups.}

Let $G = \Z_n = \Z / n \Z$.

Each elements is in its own conjugacy class.

\begin{center}
\begin{tabular}{cccc}
     & $1$ &  $t$ &  $t^{n}$\\ \hline
    $\chi_{1}$ & 1& 1&1\\
    $\chi_{2}$ &  &  & \\
    $\chi_{n}$ &  &  & \\
\end{tabular}
\end{center}

Note that the sum of the squares of the first column should be $n$, but we have  $n$ rows, so each element is  $1$. So each representation is one-dimensional.

\begin{center}\begin{tabular}{cccc}
     & $1$ &  $t$ &  $t^{n}$\\ \hline
    $\chi_{1}$ & 1& 1&1\\
    $\chi_{2}$ & 1 &  & \\
    $\chi_{n}$ & 1 &  & \\
\end{tabular}\end{center}

So we have that $\rho: G \to  GL(\C)$.
And we know even more, $\rho(t)^{n} = \rho(t^{n}) = \rho(1) = 1$, so they are roots of unity.
Let $\omega$ be the standard $n$-th root of unity, i.e. $\omega = e^{ 2 \pi i / n }$
The only possible choices for $\rho(t)$ is  $\omega^{n}$ for some $n$

\begin{center}\begin{tabular}{cccc}
     & $1$ &  $t$ &  $t^{n-1}$\\ \hline
    $\chi_{1}$ & 1& 1&1\\
    $\chi_{2}$ & 1 & $\omega^{2}$ & \\
    $\chi_{n}$ & 1 & $\omega^{n}$ & \\
\end{tabular}\end{center}

Now, the rest follows. $\chi(t^2) = \chi(t)^2$ in this case, so

\begin{center}\begin{tabular}{cccc}
     & $1$ &  $t$ &  $t^{n-1}$\\ \hline
    $\chi_{1}$ & 1& 1&1\\
    $\chi_{2}$ & 1 & $\omega^{2}$ & $(\omega^{2})^{n-1}$\\
    $\chi_{n}$ & 1 & $\omega^{n}$ & $(\omega^{n})^{n-1}$\\
\end{tabular}\end{center}

\begin{eg}
    Let $G = \Z_4$

    \begin{center}\begin{tabular}{ccccc}
         & 1 & $t$ &  $t^2$ & $t^3$ \\\hline
        $\chi_{1}$ & 1 & 1 & 1 & 1\\
        $\chi_{2}$ & $1$ &  $i$ &  $-1$ &  $-i$\\
        $\chi_{3}$ & $1$ &  $-1$ &  $1$ &  $-1$\\
        $\chi_{4}$ & $1$ &  $-i$ &  $-1$ &  $i$\\
    \end{tabular}\end{center}
    
    Note that character table can contain complex numbers, real numbers which are not integers, \ldots
\end{eg}

What if we don't have a cyclic group?


Let $A, B$ be two finite abelian groups and assume that  $\rho: A \to  \C^{\times }$ is a morphism (one-dimensional representation) and $\sigma: B \to  \C^{\times }$ is a morphism as well.
Then
\begin{align*}
    \psi: A \times B &\longrightarrow \C^{\times } \\
    (a,b) &\longmapsto \rho(a)\sigma(b)
.\end{align*}
is a morphism.
This is irreducible. Note that two ``$\psi$'s'' are the same iff the $\rho$'s and $\sigma$'s are the same. So we do get $|A\times B|$ representations.

Also, one-dimensional representation are only equivalent when they are the same.


\paragraph{Dihedral group of $4$ elements}

\[
D_4 = \{1, a, a^2, a^3, b, ab, a^2b, a^3b\} 
.\] 
And $a^4 = 1 = b^2$ and $ba = a^{-1}b$

Conjugacy classes?
\begin{align*}
    c_1 &= \{1\} \\
    c_2 &= \{a^2\}  \text{ $a^2$ is a central element}\\
    c_3 &= \{a, a^{3}\} \\
    c_4 &= \{b, a^2b\} \\
    c_5 &= \{ab, a^3b\} 
.\end{align*}

So we have
 
\begin{center}\begin{tabular}{cccccc}
     & 1 & $a^2$ & $a$ &  $b$ &  $ab$\\ \hline
    $\chi_{1}$ & $1$ & $1$ & $1$ & $1$ & $1$  \\
    $\chi_{2}$ & $?$ & $?$ & $?$ & $?$ & $?$  \\
    $\chi_{3}$ & $?$ & $?$ & $?$ & $?$ & $?$  \\
    $\chi_{4}$ & $?$ & $?$ & $?$ & $?$ & $?$  \\
    $\chi_{5}$ & $?$ & $?$ & $?$ & $?$ & $?$ 
\end{tabular}\end{center}

Now, let $n_i$ denote the elements in the first column.

$1 + n_2^2 + n_3^2 + n_4^2 + n_5^2 = 8$.
The only solution for this equation is $1, 1, 1, 1, 2$. 


\begin{center}\begin{tabular}{cccccc}
    & 1 & $a^2$ & $a$ &  $b$ &  $ab$\\ \hline
    $\chi_{1}$ & $1$ & $1$ & $1$ & $1$ & $1$  \\
    $\chi_{2}$ & $1$ & $?$ & $?$ & $?$ & $?$  \\
    $\chi_{3}$ & $1$ & $?$ & $?$ & $?$ & $?$  \\
    $\chi_{4}$ & $1$ & $?$ & $?$ & $?$ & $?$  \\
    $\chi_{5}$ & $2$ & $?$ & $?$ & $?$ & $?$ 
\end{tabular}\end{center}

A one-dimension representation is a map into $\C^{\times }$. So in particular if we mod out our group with the kernel of this representation, we have an abelian group.

So let's look for normal subgroups.
\[
N = \{1, a^2\}  \qquad  D_8 / N = \{1, \overline{a}, \overline{b}, \overline{a} \overline{b}\} \cong  \Z_2 \times \Z_2 = \left<\overline{a}, \overline{b}  \mid  \overline{a} \overline{b} = \overline{b} \overline{a}\right>
.\] 

(TODO frac other way around)

So first we want to find the representations of the Klein 4 group. We know the character table for this (abelian)
We can map $a \to  \pm 1$ and $b \to  \pm 1$ so four combinations:

\begin{center}\begin{tabular}{ccccc}\\
     & $1$ & $ \overline{a}$ & $\overline{b}$ & $\overline{a} \overline{b}$\\\hline
    $\chi_{1}$ & 1 & 1 & 1 & 1\\
    $\chi_{2}$ & 1 & 1 & $-1$ & $-1$\\
    $\chi_{3}$ & 1 & $-1$ & 1 & $-1$\\
    $\chi_{4}$ & 1 & $-1$ & $-1$ & 1\\
\end{tabular}\end{center}

So back to original table.
So let's look at $a^2$. Now $\overline{a} \mapsto  1$, so we have

\begin{center}\begin{tabular}{cccccc}
    & 1 & $a^2$ & $a$ &  $b$ &  $ab$\\ \hline
    $\chi_{1}$ & $1$ & $1$ & $1$ & $1$ & $1$  \\
    $\chi_{2}$ & $1$ & $1$ & $?$ & $?$ & $?$  \\
    $\chi_{3}$ & $1$ & $1$ & $?$ & $?$ & $?$  \\
    $\chi_{4}$ & $1$ & $1$ & $?$ & $?$ & $?$  \\
    $\chi_{5}$ & $2$ & $?$ & $?$ & $?$ & $?$ 
\end{tabular}\end{center}

We have that $a \mapsto  \overline{a}$, so 

\begin{center}\begin{tabular}{cccccc}
    & 1 & $a^2$ & $a$ &  $b$ &  $ab$\\ \hline
    $\chi_{1}$ & $1$ & $1$ & $1$ & $1$ & $1$  \\
    $\chi_{2}$ & $1$ & $1$ & $1$ & $?$ & $?$  \\
    $\chi_{3}$ & $1$ & $1$ & $-1$ & $?$ & $?$  \\
    $\chi_{4}$ & $1$ & $1$ & $-1$ & $?$ & $?$  \\
    $\chi_{5}$ & $2$ & $?$ & $?$ & $?$ & $?$ 
\end{tabular}\end{center}

Same for $b \mapsto  \overline{b}$

\begin{center}\begin{tabular}{cccccc}
    & 1 & $a^2$ & $a$ &  $b$ &  $ab$\\ \hline
    $\chi_{1}$ & $1$ & $1$ & $1$ & $1$ & $1$  \\
    $\chi_{2}$ & $1$ & $1$ & $1$ & $-1$ & $-1$  \\
    $\chi_{3}$ & $1$ & $1$ & $-1$ & $1$ & $-1$  \\
    $\chi_{4}$ & $1$ & $1$ & $-1$ & $-1$ & $1$  \\
    $\chi_{5}$ & $2$ & $?$ & $?$ & $?$ & $?$ 
\end{tabular}\end{center}

Now, what about the last one?
Multiple options:
\begin{itemize}
    \item Isometries of the square.
    \item Trick
\end{itemize}

Trick: We know that $\chi_{\rho_\text{reg}}  = \chi_{1} + \chi_{2} + \chi_{3} + \chi_{4} + 2 \chi_{5}$


\begin{center}\begin{tabular}{cccccc}
    & 1 & $a^2$ & $a$ &  $b$ &  $ab$\\ \hline
    $\chi_{1}$ & $1$ & $1$ & $1$ & $1$ & $1$  \\
    $\chi_{2}$ & $1$ & $1$ & $1$ & $-1$ & $-1$  \\
    $\chi_{3}$ & $1$ & $1$ & $-1$ & $1$ & $-1$  \\
    $\chi_{4}$ & $1$ & $1$ & $-1$ & $-1$ & $1$  \\
    $+ 2 \chi_{5}$ & $2$ & $?$ & $?$ & $?$ & $?$ \\ \hline
    $\chi_{\rho_\text{reg}}$ & 8 & 0& 0 & 0 & 0
\end{tabular}\end{center}

So we can fill in question marks:

\begin{center}\begin{tabular}{cccccc}
    & 1 & $a^2$ & $a$ &  $b$ &  $ab$\\ \hline
    $\chi_{1}$ & $1$ & $1$ & $1$ & $1$ & $1$  \\
    $\chi_{2}$ & $1$ & $1$ & $1$ & $-1$ & $-1$  \\
    $\chi_{3}$ & $1$ & $1$ & $-1$ & $1$ & $-1$  \\
    $\chi_{4}$ & $1$ & $1$ & $-1$ & $-1$ & $1$  \\
    $+ 2 \chi_{5}$ & $2$ & $-2$ & $0$ & $0$ & $0$ \\ \hline
    $\chi_{\rho_\text{reg}}$ & 8 & 0& 0 & 0 & 0
\end{tabular}\end{center}

\hr

By order of the group:

\begin{enumerate}
    \item Abelian
    \item Abelian
    \item Abelian
    \item Abelian
    \item Abelian
    \item Abelian or $ S_3 = D_3$
    \item Abelian
    \item 5 groups: 3 abelian, $ D_4$ and group of quaternions.
\end{enumerate}

So let's do the quaternions.

\[
Q = \{1, -1, i, -i, j, -j, k, -k\} 
.\] 
Conjugacy classes:

\begin{align*}
    c_1 &= \{1\} \\
    c_2 &= \{-1\} \\
    c_3 &= \{i, -i\} \\
    c_4 &= \{j, -j\} \\
    c_5 &= \{k, -k\} 
.\end{align*}

By the previous trick of group of order $8$:

\begin{center}\begin{tabular}{cccccc}
    & $1$ &  $-1$ &  $i$ &  $j$ &  $k$\\\hline
    $\chi_{1}$ & $1$ & $1$ & $1$& $1$& $1$\\
    $\chi_{2}$ & $1$ & $?$ & $?$& $?$& $?$\\
    $\chi_{3}$ & $1$ & $?$ & $?$& $?$& $?$\\
    $\chi_{4}$ & $1$ & $?$ & $?$& $?$& $?$\\
    $\chi_{5}$ & $2$ & $?$ & $?$& $?$& $?$\\
\end{tabular}\end{center}

Normal subgroups of $Q$? Every subgroup is normal.
 \[
     \{1, -1\}  \triangleleft Q \implies \frac{Q}{\{1, -1\} } = \overline{1}, \overline{i}, \overline{j}, \overline{k} = \Z_2 \times \Z_{2} = \left<\overline{j}, \overline{i}  \mid \overline{i} \overline{j} = \overline{j} \overline{i}\right>
.\] 

So:

\begin{center}\begin{tabular}{cccccc}
    & $1$ &  $-1$ &  $i$ &  $j$ &  $k$\\\hline
    $\chi_{1}$ & $1$ & $1$ & $1$& $1$& $1$\\
    $\chi_{2}$ & $1$ & $1$ & $1$& $-1$& $-1$\\
    $\chi_{3}$ & $1$ & $1$ & $-1$& $1$& $-1$\\
    $\chi_{4}$ & $1$ & $1$ & $-1$& $-1$& $1$\\
    $\chi_{5}$ & $2$ & $?$ & $?$& $?$& $?$\\
\end{tabular}\end{center}

We can do the same thing as before to find the last ones.

\begin{center}\begin{tabular}{cccccc}
    & $1$ &  $-1$ &  $i$ &  $j$ &  $k$\\\hline
    $\chi_{1}$ & $1$ & $1$ & $1$& $1$& $1$\\
    $\chi_{2}$ & $1$ & $1$ & $1$& $-1$& $-1$\\
    $\chi_{3}$ & $1$ & $1$ & $-1$& $1$& $-1$\\
    $\chi_{4}$ & $1$ & $1$ & $-1$& $-1$& $1$\\
    $\chi_{5}$ & $2$ & $-2$ & $0$& $0$& $0$\\
\end{tabular}\end{center}

So character tables of $Q$ and  $ D_4$ are the same. But $Q \not \cong D_4$!

An explicit representation of the last one:

\[
    \rho(1) = \begin{pmatrix}
        1 & 0 \\
        0 & 1
    \end{pmatrix} \qquad
    \rho(-1) = \begin{pmatrix}
        -1 & 0 \\
        0 & -1
    \end{pmatrix} \qquad
    \rho(i) = \begin{pmatrix}
        i & 0 \\
        0 & -i
    \end{pmatrix} \qquad
    \rho(j) = \begin{pmatrix}
        0 & 1 \\
        -1 & 0
    \end{pmatrix} \qquad
    \rho(k) = \begin{pmatrix}
        0 & i \\
        i & 0
    \end{pmatrix}
.\] 

TIP: atlas of finite groups.
