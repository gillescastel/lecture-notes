\lesson{3}{di 06 okt 2020 11:25}{}


Recap.
Let 

\begin{align*}
    \rho: G \to \GL_{k}(V) && \sigma: G \to \GL_{k}(W)
.\end{align*}
Then define
\[
\psi: \to \GL_{k}(\Hom_{k}(V, W))
\] 
as
\[
    \psi(g) : \Hom_{k}(V, W) \to  \Hom_{k}(V, W) : f \mapsto \psi(g) (f)
,\] 
where

\[
    \psi(g)(f) : V \to  W : v \mapsto  \sigma(g) f(\rho(g)^{-1} w)
.\] 
The map $\psi(g)$ is linear map (check this).
To check that is a representation, we check the following:
\begin{align*}
    (gh) \cdot  f (v) &= (gh) \cdot  f((gh)^{-1} \cdot v)\\
                      &= g \cdot  h \cdot  f ( h^{-1} \cdot  g ^{-1} \cdot  v )\\
                      &= g \cdot  h \cdot  f(v)
.\end{align*}

\begin{remark}
    What is $(\Hom_{k}(V, W))^{G} =: \Hom_{G}(V, W)$?

    $f \in \Hom_{G}(V, W)$ if $g \cdot  f = f$ for all $g \in G$.
    So
    \begin{align*}
        \sigma(g) f(\rho(g)^{-1} v) &= f(v) \quad \forall  g \in  G, v \in V\\
        f(\rho(g^{-1}) v) &= \sigma(g^{-1} ) f(v)\\
        f(\rho(g) v) &= \sigma(g) f(v)
    .\end{align*}
    This is exactly the definition of a $G$-map.
    So 
    \[
    f \in \Hom_{k}(V, W)^{G} \iff \text{ $f $ is a $G$-map }
    .\] 

\end{remark}

\paragraph{Special case: contragrediant representation}
    
    Consider $\rho: G \to \GL_{k}(V)$ and let $k$ be the trivial $1$-dimensional $G$-module.
    (I.e. $\sigma: G \to \GL_{k}(k): g \mapsto  \sigma(g) = k \to  k : x \mapsto x$)

    So the dual space $V^{*} = \Hom_{k}(V, k)$ becomes a $G$-module with $\psi: G \to \GL_{k}(V^{*})$ with
    \[
        \psi(g)(f)(v) = \sigma(g) f(\rho(g)^{-1} \sigma) = f(\rho(g)^{-1} \sigma)
    ,\] 
    as $\sigma(g)$ is doing nothing.
    We write $\rho^{*} := \psi$.
    This is called the contragrediant representation.


Let $\mathbf{v} = \{ v_1, \ldots, v_n\} $ be a basis of $V$ and let $\mathbf{v}^{*} = \{v_1^{*}, \ldots, v_n^{*}\} $ the dual basis:
\[
v_i^{*} : V \to  k : v_j \mapsto \delta_{ij}
.\] 

Let $[\rho(g)]_{\mathbf{v}} = r_{ij}(g)$.
Then $[\rho^{*}(g)]_{\mathbf{v}^{*}} = r_{ji}(g^{-1}) = r_{ij}(g^{-1})^{T}$

\begin{proof}
    Let $[\rho^{*}(g)]_{\mathbf{v}} = t_{ij}(g)$.
    Then
    \[
        \rho^{*}(g)(v^{*}_j) = \sum_{k=1}^{m} t_{kj}(g) v_k^{*}
    \]
    So
    \[
        \rho^{*}(g)(v_j^{*})(v_i) = \sum_{k=1}^{m} t_{kj}(g) v_k^{*}(v_i) = t_{ij}(g)
    .\] 

    On the other hand:
    \begin{align*}
        \rho^{*}(g)(v_j^*)(v_i) :&= v_j^* ( \rho(g^{-1}) v_i )\\
                                 &= v_j^* (\sum r_{ki}(g^{-1}) v_k)\\
                                 &= r_{ji}(g^{-1})
    .\end{align*} 
    This finishes the proof.
\end{proof}

\begin{prop}
    $\Hom_{k}(V, W) \cong W \otimes V^*$ as $G$-modules.
\end{prop}
\begin{proof}
    Exercise
\end{proof}

\nsection{2}{4}{Permutation representations (2.4 -- 2.6)}

Let $G$ be a finite group.
Let $X$ be a finite set. 

We assume that an action of $G$ on $X$ is given, i.e.
\[G \times  X \to  X: (g, x) \mapsto  g\cdot x,\]
with
\begin{enumerate}[(1)]
    \item $1x = x$
    \item $g(hx) = (gh) x$
\end{enumerate}

In other words, this is group homomorphism $G \to  SX$.

We can use this to build a linear representation!
So first we need to find a vector space.
Let $X = \{x_1, \ldots, x_n\} $
Then we choose as our vector space
\[
    k[X] = \{k_1 x_1 + k_2 x_2 + \cdots + k_n x_n  \mid  k_i \in k\} 
.\] 
This is a vector space with basis $X$.

\begin{remark}
    You can also see this as $k[X] = \operatorname{Map}(X, k)$.
\end{remark}

Now we can extend the action of $G$ linearly onto our vector space.
Define 
\[
    \rho: G \to \GL_{k}(k[X]): \rho(g)(k_i x_i) = k_i (g\cdot x_i)
.\] 
We call this the permutation representation.

\begin{ex}
    Let $G = S_4$ which acts on $X = \{e_1, e_2, e_3, e_4\}$.
    Then
    \[
        \rho(\operatorname{Id}) = \begin{pmatrix}
            1 & 0 & 0 & 0\\
            0 & 1 & 0 & 0\\
            0 & 0 & 1 & 0 \\
            0 & 0 & 0 & 1
        \end{pmatrix}
        \qquad
        \qquad
        \rho(23) = \begin{pmatrix}
            1 & 0 & 0 & 0\\
            0 & 0 & 1 & 0\\
            0 & 1 & 0 & 0 \\
            0 & 0 & 0 & 1
        \end{pmatrix}
    .\] 

    \[
        \rho((1,2)(3,4)) = \begin{pmatrix}
            0 & 1 & 0 & 0\\
            1 & 0 & 0 & 0\\
            0 & 0 & 0 & 1 \\
            0 & 0 & 1 & 0
        \end{pmatrix}
        \qquad
        \qquad
        \rho(1,2,3) = \begin{pmatrix}
            0 & 0 & 1 & 0\\
            1 & 0 & 0 & 0\\
            0 & 1 & 0 & 0 \\
            0 & 0 & 0 & 1
        \end{pmatrix}
    .\] 

    We see that we get a permutation matrix.
\end{ex}

\begin{remark}
    Note that $\operatorname{Tr}(\rho(g))$ is the number of fixed points, i.e. $|X^{g}|$
\end{remark}


\begin{eg}[Important example: regular presentation]
    Let $X = G$, so $G$ acts on itself by left multiplication.
    \[
        G \times G : (g, h) \mapsto  g\cdot h = gh
    .\] 
    Elements of $k[G]$ are of the form  $\sum_{g \in G} \alpha_g g$ with $\alpha_g \in k $.
    And so for the corresponding premutation representation we have that
    \[
        \rho(g)(\sum \alpha_h h) = \sum \alpha_h(gh)
    .\] 
    This representation is called the regular representation. 
    Notation: $\rho_\text{reg}$.

    (This is important because when we decomse $\rho_\text{reg} $ into irreducible representations, we get \emph{all} possible irreducible representations. See later)

    What about the trace?

    \[
        \operatorname{Tr}(\rho_\text{reg} (g)) = |G^{g}| = \begin{cases}
            |G| &\text{if $g = 1$ }\\
            0 &\text{if $g \neq 1$}
        \end{cases}
    .\] 
\end{eg}
\begin{remark}
    Note that in the preceding example, $k[G]$ is a ring!
    We can add. What about multiplication?

    \[
        (\sum_{g \in G} \alpha_g g)(\sum_{h \in G} \beta_h h)
        = \sum_{g \in G} \sum_{h \in G} \alpha g \beta_h (gh)
        = \sum_{g \in G} \left(\sum_{k \in G} \alpha_{k} \beta_{k^{-1} g} \right)g 
    ,\] 
    where in the last equality, we grouped them for each $g \in G.$

    This is ring with identity: the one is $1 e_g$.

    We call this the group ring or group algebra!
\end{remark}

\begin{remark}
    If $V$ is a $G$-module via $\rho: G \to \GL_{k}(V)$.
    Then we can turn $V$ into a $k[G]$-module by defining
    \[
        \left(\sum_{g \in G} \alpha_g g\right) v = \sum_{g \in G} \alpha_g ( gv )
        := \sum_{g \in G} \alpha_g (\rho_g v)
    .\] 

    We can also go the other way around.  (Careful: we need ask that $1$  acts as the identity (?))
\end{remark}

\nchapter{3}{Character theory}

Let $G$ be a finite group and $k = \C$.

\nsection{3}{1}{Characters and class functions of a finite group}

\begin{definition}
    Let $\rho: G \to \GL(V)$ be a representation.
    Then the character of $\rho$ is
     \[
         \chi_\rho: G \to  \C: g \mapsto  \operatorname{Tr}(\rho(g))
    .\] 
\end{definition}

\begin{eg}
    Remember that for a permutation representation, this is $|X^{g}|$.
\end{eg}

Characters are class functions:

\begin{definition}[Class function]
    A function $\theta: G \to  \C$ is a class function iff 
    \[
        \theta(g) = \theta(h g h^{-1})  \qquad \forall  g, h \in G
    .\] 
\end{definition}

Indeed,
\[
    \Tr(\rho(hgh^{-1})) = \Tr(\rho(h) \rho(g) \rho(h^{-1})) = \Tr(\rho(g))
,\] 
so that $\chi$ is a class function.

\begin{remark}
    Remember that two permutations are conjugate iff the have the same type of disjoint cycle decomposition. (Same number of $n$-cycles for all $n$).
\end{remark}

\begin{theorem}[3.5]
    Let $\rho: G \to \GL(V)$ be a representation.
    Fix $g \in G$.
    Then there exists a basis $\mathbf{v} = \{v_1, \ldots, v_n\} $ of $V$ w.r.t.\ which $[\rho(g)]_{\mathbf{v}}$ is a diagonal matrix.
\end{theorem}
So we can find a basis of eigenvectors.

\begin{proof}
    The order of $g$ is finite.
    So $g^{d} = 1$ for some $d \in \N$.
    So $(\rho(g))^{d} = I$. So $\rho(g)$ is a zero of  $X^{d} - 1$.
    This implies that the minimal polynomial of $\rho(g) $ divides  $X^{d} - 1$.
    This polynomial splits into linear factors.
    So the minimal polynomial of $\rho(g)$  also splits into unique (different) linear factors.
    This implies that $\rho(g)$ can be put into diagonal form.
\end{proof}

So 
\[
    [\rho(g)]_{\mathbf{v}} = \begin{pmatrix}
        \lambda_1 & 0 & 0\\
        0 & \ddots & 0\\
        0 & 0 & \lambda_n
    \end{pmatrix}
.\] 

But we also know that because $[\rho_(g)]^{d} = 1$, we have $\lambda^{d} = 1$ as well.  So $\lambda_i$ are roots of unity!
