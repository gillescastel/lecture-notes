\lesson{22}{di 08 dec 2020 13:05}{}


\begin{corollary}
    If $A$ is a module ``divisible by'' $n= |G|$, so for any $a \in A$, there exists a $b \in A$ such that $nb = a$.
    Then $H^{i}(G, A) = 0$ for all $i>0$
\end{corollary}
\begin{eg}
    If $A$ is a vector space over $\Q$, then this is satisfied!
\end{eg}
\begin{proof}
    For all $f \in Z^{i}(G, A)$, we have $nf = d \lambda$.
    Then  $f = \frac{1}{n} d\lambda$.
    So $f \in B^{i}(G, A)$.
\end{proof}
\begin{remark}[Exam, problems]
    This is something we will need to use in the problem on the alternative proof of Maschkes theorem.
\end{remark}

\begin{remark}
    We were talking about unnormalized bar resolution.
    When considering the normalized bar resolution, we only consider map that are normalized.
    Normalized cocycles $Z^{2}$: so $f(1,x) = f(x,1) = 0$.
    Normalized coboundaries $B^{2}$: $g(1) = 1$.
\end{remark}

\nsection36{Group extensions and $H^{2}$ }
\begin{definition}
    A group extension is a short exact sequence of groups.
    \[
    0 \to  A \to  E \xrightarrow{\pi}   G \to  1
    ,\] 
    where $A$ is Abelian\footnote{This is why we use 0 for the trivial group on the left and on the right  $1$, but they're really the same groups} 
    We will often see $A$ as a subgroup of  $E$.
\end{definition}

\begin{remark}
    Recall that if $A$ is a  $G$-module, then $E = A \rtimes G$ is a group extension as follows:
    \[
    0 \to  A \to  A \rtimes G \to  G \to  1
    .\] 

\end{remark}

Claim: if  we have a group extension, then there is a $G$-module structure on $A$. 
Chose a section $\sigma: G \to  E$ such that $\pi  \circ  \sigma = 1_G$.
Then define $g\cdot a = \sigma(g) a \sigma(g)^{-1}$
Note that $g \cdot a$ is independent on the chosen $\sigma$.
Why? Change $\sigma$ to  $\sigma'$, then they will differ by an element of  $a$, but  $A$ is abelian. (TODO fix this)

Exercise: this defines a morphism  $G \to  \Aut A$.


Doing this for our example of $\rtimes$, we have
 \[
     (0,g) (a,1) (0,g)^{-1} = \cdots = (ga,1)
,\] 
so action defined in semi-direct product is the same action like from $\sigma$.

For any $G$-module $A$, we do have a group extension, but there might be more than one.
How do we classify group extensions?

\begin{definition}
    Two extensions are equivalent if there exists a morphism $\phi: E_1 \to  E_2$ such that the following commutes:

    \[
        \begin{tikzcd}
            0 \arrow[r, ""] 
            & A \arrow[r, ""] \arrow[d, "\operatorname{Id}"]
            & E_1 \arrow[r, ""] \arrow[d, "\phi"]
            & G \arrow[r, ""] \arrow[d, "\operatorname{Id}"]
            & 1 \\
            0 \arrow[r, ""] 
            & A \arrow[r, ""]
            & E_2 \arrow[r, ""]
            & G \arrow[r, ""]
            & 1 \\
        \end{tikzcd}
    \]
    If this commutes, this will automatically be an isomorphism.
\end{definition}

\begin{prop}
Equivalent extensions induce the same $G$-module structure on $A$
\end{prop}
\begin{proof}
    Choose section $\sigma: G \to  E_1$ on diagram above.
    Then there exists a section of the second row: $\phi  \circ  \sigma: G \to  E_2$.
    More or less trivial to prove that these induce same $G$-module structure.
\end{proof}

Question: Can we classify extensions inducing a given $G$-module structure on $A$ up to equivalence?

\begin{definition}
    $\epsilon(G, A)$ is the set of extensions inducing the given  $G$-module structure up to equivalence.
\end{definition}

Main theorem:
\begin{theorem}
    There is a bijection between $\epsilon(G, A)$ and  $H^2(G, A)$.
\end{theorem}

\paragraph{Step 1}
Consider $0 \to  A \to  E \to  G \to 0$.
Given an element of $\epsilon(G, A)$, we want to find a  $f \in Z^2(G, A)$.

Choose $\sigma: G \to  E$ a section.
Then 
\[
    \sigma(xy) \neq \sigma(x) \sigma(y)
\] 
in general, because sections are only set maps, and not group morphisms. Recall if it is a group morphism it is called a splitting, and not very interesting.
However, when applying $\pi: E \to  G$, we get
\[
    \pi(\sigma(xy)) = \pi(\sigma(x) \sigma(y))
,\] 
so $a\sigma(xy) = \sigma(x)\sigma(y)$, with  $a \in  \ker \pi = A$.
This $a$ of course depends on  $x$ and  $y$, so we have a map  $G^2 \to  A$.
We claim that this is $f$.
This also depends on the section.
\begin{definition}
    Define $f: G^2 \to  A$ by $f(x,y) \sigma(xy) = \sigma(x)\sigma(y)$
\end{definition}

Now, we prove that $f$ is a cocycle.

We know that
\[
    \sigma((xy)z)  = \sigma(x(yz))
,\] 
and
\[
    \sigma(xy) = f^{-1}(x,y) \sigma(x) \sigma(y)
.\] 
Applying this twice to $\sigma((xy)z) $ and  $\sigma(x(yz))$,
 \[
     f^{-1}(xy,z) f^{-1}(x,y) \sigma(x)\sigma(y)\sigma(z) = f^{-1}(x,yz) \sigma(x) f^{-1}(y,z) \sigma(y) \sigma(z)
.\] 
We would like to get red of $\sigma$'s, so we want them all to the right. Okay on the left, need to fix on the right.

\begin{align*}
    f^{-1}(xy,z) f^{-1}(x,y) \sigma(x)\sigma(y)\sigma(z) &= f^{-1}(x,yz) \sigma(x) f^{-1}(y,z) \sigma(x)^{-1} \sigma(x)\sigma(y) \sigma(z)\\
    f^{-1}(xy,z) f^{-1}(x,y) &= f^{-1}(x,yz) \underbrace{\sigma(x) f^{-1}(y,z) \sigma(x)^{-1}}_{\in A}
\end{align*}
Underbrace is action of $x$ on  $f(yz)$
Swapping to additive notation, we have
\[
    -f(xy,z) - f(x,y) = -f(x,yz) - x f(y,z)
,\] 
or
\[
    xf(y,z) - f(xy,z) + f(x,yz) - f(x,y) = 0
,\] 
which is exactly the 2-cocycle condition.
So $f$ is a 2 cocycle.


Okay, everything alright, except that $f$ does depend on the chosen  $\sigma$! 

\paragraph{Step 2}

What happens if we chose another section, $\sigma' : G \to  E$?
Then we know that $\sigma'(x) = \sigma(x)$ up to an element of  $a$, because when we project down, we should get the same thing! We write this as
 \[
     \sigma'(x) = \beta(x) \sigma(x) \qquad \beta(x) \in A
.\] 

\begin{align*}
    f'(x,y) &= \sigma'(x) \sigma'(y) \sigma'(xy)^{-1}\\
            &= \beta(x) \sigma(x) \beta(y) \sigma(y) (\beta(xy) \sigma(xy))^{-1}\\
            &= \beta(x) \sigma(x) \beta(y) \underbrace{\sigma(x)^{-1} \sigma(x)}_{1} \sigma(y) \sigma(xy) ^{-1} \beta(xy)^{-1}\\
            &= \beta(x) \underbrace{\sigma(x) \beta(y) \sigma(x)^{-1}}_{A} \underbrace{\sigma(x) \sigma(y) \sigma(xy) ^{-1}}_{f(x,y)} \beta(xy)^{-1}
\end{align*}
Going again to additive notation,

\[
    f'(x,y) = \beta(x) + x \beta(y) + f(x,y) - \beta(xy)
,\] 
or
\[
    f'(x,y) -f(x,y) = x\beta(y) - \beta(xy) + \beta(x) = (d\beta)(x,y)
,\] 
so $f' - f \in B^2(G, A)$

Conclusion: $f + B^2(G, A)$ is well defined.
So we do have a map which attaches to any extensions in a unique way an element of $H^2(G, A)$.
\paragraph{Step 3}
Equivalent extensions give rise to same cohomology class.

\[
    \begin{tikzcd}
        0 \arrow[r, ""] 
        & A \arrow[r, ""] \arrow[d, "\operatorname{Id}"]
        & E_1 \arrow[r, ""] \arrow[d, "\phi"]
        & G \arrow[r, ""] \arrow[d, "\operatorname{Id}"] \arrow[l, "\sigma", bend right, swap]
        & 1 \\
        0 \arrow[r, ""] 
        & A \arrow[r, ""]
        & E_2 \arrow[r, ""]
        & G \arrow[r, ""] \arrow[l, "\sigma' = \phi  \circ  \sigma", bend left]
        & 1 \\
    \end{tikzcd}
\]

Define $f$ from upper row.
Then 
\[
    f(x,y) \sigma(x,y) = \sigma(x) \sigma(y)
.\] 
Then apply $\phi$
\[
    \phi(f(x,y)) \phi\sigma(x,y) = \phi\sigma(x) \phi\sigma(y)
,\] 
so
\[
    \phi f(x,y) \sigma'(x,y) = \sigma'(x) \sigma'(y)
.\] 
The left hand side is the 2 cocycle attached to second extension.
But $f(x,y) \in A$, so looking at the diagram, applying $\phi$, we get that  $\phi(f(x,y)) = f(x,y)$.

So
 \[
     f(x,y) \sigma'(xy) = \sigma'(x) \sigma'(y)
.\] 

Conclusion: we have a map
\[
    \epsilon(G, A) \to  H^2(G, A)
.\] 

Now, prove that this is bijective map.
\paragraph{Injective.}
Good exercise!
How?
Assume you have two extensions:

\[
    \begin{tikzcd}
        0 \arrow[r, ""] 
        & A \arrow[r, ""] \arrow[d, "\operatorname{Id}"]
        & E_1 \arrow[r, ""] 
        & G \arrow[r, ""] \arrow[d, "\operatorname{Id}"] \arrow[l, "\sigma", bend right, swap]
        & 1 \\
        0 \arrow[r, ""] 
        & A \arrow[r, ""]
        & E_2 \arrow[r, ""]
        & G \arrow[r, ""] \arrow[l, "\sigma'", bend left]
        & 1 \\
    \end{tikzcd}
\]
and let $f$ be the cocycle to first and $f'$ to the second.

And we know that  $f - f'$ is a coboundary.
Then we can change section such that corresponding cocycles are really the same(?)
Then we can define  $\phi: E_1 \to  E_2$
Any element of $E$ is of the form  $a\sigma(g)$, for unique  $a$ and unique  $g$.
Then 
 \[
     \phi(a\sigma(g)) = a \sigma'(g)
.\] 
Now check it's an isomorphism.
More information in video.

\paragraph{Surjective.}

Let $f \in Z^2(G,A)$.
Let $E = A \times G$ with product
\[
    (a_1,g_1)(a_1,g_2) = (a_1+g_1a_2+\boxed{f(g_1,g_2)}, g_1g_2)
.\] 
Prove that this is group. You will need cocycle condition.
Then we have an extension:

\[
 0\to A \to  E_f \to  G \to  0
.\] 
You can check that this works.

\paragraph{Conclusion}
\[
    \epsilon(G,A_) = H^2(G,A)
.\] 

\begin{remark}
    The extension you end up with the semidirect product gives you the zero element in $H^2(G,A)$.
\end{remark}
\begin{remark}
    Extensions which are equivalent with the semidirect product are called split extensions.
\end{remark}
\begin{definition}
    An extension  
    \[
    0 \to  A \to  E \to G \to  1
    \] 
    is split iff there exists a morphism $\sigma: G \to  E$ such that $\pi  \circ  \sigma = 1_G$,
    or equivalently, the following commutes

    \[
        \begin{tikzcd}
            0 \arrow[r, ""] 
            & A \arrow[r, ""] \arrow[d, "\operatorname{Id}"]
            & E \arrow[r, ""]  \arrow[d, "\phi"]
            & G \arrow[r, ""] \arrow[d, "\operatorname{Id}"] 
            & 1 \\
            0 \arrow[r, ""] 
            & A \arrow[r, ""]
            & A \rtimes G \arrow[r, ""]
            & G \arrow[r, ""] 
            & 1 \\
        \end{tikzcd}
    \]
\end{definition}

\begin{remark}
    Exercise 6.6.1 is wrong!!!!
\end{remark}
