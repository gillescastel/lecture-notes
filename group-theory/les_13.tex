\lesson{13}{di 10 nov 2020 10:57}{}

Fact:

\begin{prop}
    If $G$ is a group and $N \triangleleft G$.
    Then $G$ is solvable iff $N$ and  $\frac{G}{N}$ are solvable.
\end{prop}
\begin{proof}
    Not hard, try this yourself!
\end{proof}

This statement does not hold for nilpotent groups!

\begin{prop}
    If $G$ is nilpotent, then $N$ and $\frac{G}{N}$ are nilpotent. The converse is not true
\end{prop}
\begin{eg}
    Consider $D_n$ and subgroup $\left<a \right>$.
    Then $\left<a \right>$ and $\frac{D_n}{\left<a \right>} = \Z_2$ are nilpotent, but $D_n$ is not when $n \neq 2^{k}$
\end{eg}

\begin{theorem}
    A solvable group $G$ has max iff $G$ is polycyclic.
\end{theorem}
\begin{proof}
    Not so difficult to prove.

    If it has max. 
    Look at
    \[
        1 \le  G^{(d)} \le  \cdots < G^{(1)} \le G
    .\] 
    If group has max, then these subgroups are fg. These quotients are abelian.
    So quotients are fg abelian. So polycyclic (?)
\end{proof}



\begin{prop}
    If $N$ is torsion free and $\frac{G}{N}$ is torsionfree, then $G$ is torsionfree.
\end{prop}
\begin{proof}
    Assume $g^{k}=1$. Then $\overline{g} \in G / N$ and $\overline{g}^{k} = 1$, so $\overline{g} = 1$, so $g \in N$. But $N$ is itself is torsionfree, so $g = 1$.
\end{proof}

\begin{remark}
    So poly-$C_{\infty}$ implies also poly-torsionfree. And this is also implies torsionfree.
    So any poly-$C_{\infty}$-by-finite group has a subgroup of finite index which is torsionfree.
\end{remark}

\section*{B. Nilpotent groups}
Nothing new. Will not be asked on the exam, execpt bottom part on page $50$.

\begin{definition}[Fitting subgroup]
    Let $G$ be any group, then \[
    \Fit (G) = \left<N  \mid  N \triangleleft G \text{ and $N$ is nilpotent} \right>
\]
\end{definition}

\begin{corollary}[13]
    If $G$ is polycyclic-by-finite (or more generally if $G$ has max), then $\Fit G$ is nilpotent.
    So $\Fit G $ is then the maximal normal nilpotent subgroup of  $G$.
\end{corollary}
\begin{proof}
    $\Fit G$ is fg, because we're working in a group which has max.
    So
    \[
    \Fit G = \left<n_1, \cdots, n_k \right>
    .\] 
    with $n_i = a_1 a_2 \cdots a_k$ with $a_i \in N_k$ where $N_k$ are normal nilpotent subgroups.
    So $\Fit G$ is generated by a finite number of nilpotent normal subgroups. (finitely elements and all of them can be written using elements of finitely man $N_k$'s)
    Using the fact that  $N M$ is nilpotent when  $N\triangleleft G$ and $M \triangleleft G$ are nilpotent,
    we find that $\Fit G$ is nilpotent.
\end{proof}

\begin{remark}
    If you have infinite polycyclic-by-finite group, this group will also be infinite.
    So you get a non-trivial nice nilpotent normal subgroup.
\end{remark}

\section*{C. Some theorems about polycyclic groups.}

Techniques for using things using polycyclic groups
\begin{itemize}
    \item Induction on the length of the series.
    \item Induction on the Hirsch length (number).
\end{itemize}
\begin{definition}[Hirsch length]
    If $G$ is polycyclic-by-finite, then
    \[
        h(G) = \text{number of infinite cyclic factors in a series of subgroups}
    .\] 
    So series
    \[
    G_0 = 1 \triangleleft  G_1 \triangleleft \cdots \triangleleft  G_n   = G
    .\] 
    such  that $\frac{G_{i+1}}{G_i}$ is $C_\infty$ or finite.
    Count number of times it is $C_{\infty}$.
\end{definition}

We need to prove that this number does not depend on the series.
We prove this as follows: when refining a sequence, the new quotients are $C_{\infty}$ and finite.
And second part is then proving that given two sequences, we can refine them both such that subgroups match up.




\begin{prop}
    Assume $H\le G$, $N \triangleleft G$, $G$ is polycyclic-by-finite.
    \begin{enumerate}[(1)]
        \item  $h(H) \le h(G)$.
        \item $h(H) = h(G)$ holds iff $|G:H| < \infty$.
        \item $h(G) = h(N) + h(\frac{G}{N})$
        \item $h(G) = 0$ iff  $G$ is finite. (follows from $(2)$)
    \end{enumerate}
\end{prop}

So to do induction on Hirsch length, we need to finite infinite normal subgroups!

\begin{lemma}[6]
    If $G$ is an infinite polycyclic-by-finite group, then $G$ has a normal subgroup $A \triangleleft G$ with $A \cong \Z^{k}$ where $k \in \N_0$.
\end{lemma}
\begin{proof}
    $G$ is polycyclic-by-finite, so also  (poly-$C_{\infty}$)-by-finite.
    We know that poly-$C_\infty$ implies torsionfree and poly-about so solvable.
    Denote with $S$ this poly-$C_{\infty}$ group.
    So $S \triangleleft _f G$ torsionfree and solvable.
   Solvable so $S ^{(d)} \neq 1$ and $  {(d+1)} = 1$.
   Now, $S^{(d)}$ is abelian,  because $[S^{(d+1)} = [S^{(d)}, S^{(d)}] = 1$.
   Also a subgroup of a torsionfree group, so also torsionfree.
   And fg because subgroup of group which has max (because $G$ polycyclic-by-finite).
   So $S^{(d)} \cong \Z^{k}$.

   Why is $S^{(d)}$ normal in $G$?
   Note that $S^{(d)}$ is characteristic in $S$. (Any automorphism in $S$ maps this subgroup to itself)
   And $S \triangleleft G$, so the conjugation action is such an automorphism than can be restricted to $S$, so it maps $S^{(d)}$ to itself. So normal.

   Q: Why is $S^{(d)}$ still infinite? Because it is torsionfree.
\end{proof}

\begin{definition}[residually]
    Let $G$ be a group. Then we say that $G$ is residually $P$ iff for any  $g \in G \setminus \{1\} $, there exists a $N \triangleleft G$ such that $g \not\in N$ and $\frac{G}{N}$ has property $P$. 
    \[
    \forall g \in G \setminus \{1\} : \exists N \triangleleft G : g \not\in N \text{ and } \frac{G}{N} \text{ has $P$}
    .\] 
    Or in other words:
    \[
        \forall g \in G \setminus \{1\}  \exists N \triangleleft G : \frac{G}{N} \text{ has $P$ and $\overline{g} \neq 1$ }
    .\] 
    Any element of $G$ survives in a quotient having property $P$.

    Equivalently:
    \[
    \bigcap \left\{N  \mid  N\triangleleft G, \frac{G}{N} \text{ has $P$}\right\}  = 1
    .\] 
\end{definition}

\begin{theorem}[1, big result!]
    Any polycyclic-by-finite group is residually finite.
\end{theorem}

Use of this theorem: You can study any polycyclic-by-finite group by looking all of its finite quotients.
Because any element survives a finite quotient.
\begin{eg}
    To prove that two elements are different, we can use the fact that we should always find a finite quotient in which they are different.
\end{eg}

\begin{proof}
    By induction on the Hirsch length of $G$.

    \begin{itemize}
        \item If $h(G) =0$, then  $G$ is finite. Consider $N=1 \triangleleft G$ and $\frac{G}{N}$ finite. $ \bigcap N = 1 $.
        \item Now assume that $h(G) > 0$ and the theorem holds for groups of smaller Hirsch length.
            Because $h(G) > 0$,  $G$ is infinite, so there exists a  $A \cong \Z^{k} \triangleleft G$ with $k>0$.
            Now consider $A^{m} = \{a^{m}  \mid  a \in A\} = (m\Z)^{k} $.
            Note that $|A : A^{m}| = m^{k} < \infty$
            So $h(A) = h(A^{m}) = k > 0$.
            Consider $\frac{G}{A^{n}}$. Note $A^{m} \triangleleft G$ (Again: $A^{m}$ is char subgroup of $A$ and $A$ is normal).
            We have
            \begin{align*}
                h(G) &= h(A^{m}) + h\left(\frac{G}{A^{m}}\right)\\
                     &= k + h\left(\frac{G}{A^{m}}\right)
            .\end{align*} 
            So $h(\frac{G}{A^{m}}) < h(G)$ (Strictly because $k > 0$)

            Now, using the induction hypothesis. 
            So
            \[
                \bigcap \left\{\frac{N}{A^{m}} \mid A^{m} \triangleleft  N \triangleleft  G , \frac{G / A^{m}}{N / A^{m}} = \text{ finite}\right\}  = \frac{A^{m}}{A^{m}} = 1
            .\] 
            Or in other words:

            \[
                \bigcap \left\{\frac{N}{A^{m}} \mid A^{m} \triangleleft  N \triangleleft  G , \frac{G }{N} = \text{ finite}\right\}  =
                \bigcap \left\{\frac{N}{A^{m}} \mid A^{m} \triangleleft  N \triangleleft_f  G\right\}  = \frac{A^{m}}{A^{m}}
            .\] 
            In other words:
            \[
                \bigcap \left\{N \mid A^{m} \triangleleft  N \triangleleft_f  G\right\}  = A^{m}
            .\] 

            Fact: 
            \[
                \bigcap_{m \in \N_0} A^{m} = (0, 0, \cdots 0) = 1 \qquad \text{ because $\bigcap m \Z = 0 $}
            .\] 
            So
            \[
            \bigcap \{N  \mid  N \triangleleft_f G\}  \le  \bigcap_{m \in \N_0} \{N  \mid A^{m} \triangleleft  N_f \triangleleft G\}  
            ,\] 
            because intersection set is bigger on the left.
            But
            \[
            \bigcap_{m \in \N_0} \{N  \mid A^{m} \triangleleft  N_f \triangleleft G\}   = 
            \bigcap_{m \in \N_0} A^{m} = 1
            .\] 
    \end{itemize}

\end{proof}

Next result: you can discover if a pcbf group is nilpotent by only looking at it's finite quotients.


\begin{theorem}[2]
    Let $G$ be a polycylic-by-finite group, then $G$ is nilpotent iff every finite quotient of $G$ is nilpotent.
\end{theorem}
\begin{proof}
    (More details than in the book)
    If $G$ nilpotent, then any quotient of $G$ is nilpotent.
    So the only non-trivial part is the converse statment.

    By induction on $h(G)$.
     \begin{itemize}
         \item $h(G) = 0$, then  $G$ is finite, so $\frac{G}{1}$ is a finite quotient of itself, so the theorem holds.
         \item $h(G) > 1$. Then there exists $A \triangleleft G$ with $A \cong \Z^{k}$ and $k > 0$.
             Consider  $A^{p}$ with $p$ a prime number.
             We know that $A^{p} \triangleleft G$.

             Claim: $\frac{G}{A^{p}}$ satisfies conditions of theorem. In the sense that any nilpotent quotient of this group is nilpotent.
             Because a finite quotient of $\frac{G}{A^{p}}$ is of the form 
             \[
                 \frac{ G / A^{p}}{N / A^{p}} \qquad \text{with $A^{p} \triangleleft  N \triangleleft_f G$}
             .\] 
             But  this is isomorphic to $\frac{G}{N}$, which is a finite quotient of $G$, so nilpotent.

             So assuming that every finite quotient of $G$ is nilpotent, we were able to prove that every finite quotient of $\frac{G}{A^{p}}$ is nilpotent. 

             We have $h(G) = h(A^{p}) + h(\frac{G}{A^{p}})$, so $h(\frac{G}{A^{p}}) < h(G)$, so we can use the induction hypothesis.
             So $\frac{G}{A^{p}}$ is nilpotent.
             This holds for any $p$.

             So $\frac{G}{A^{p}}$ is nilpotent. So $\gamma_m (\frac{G}{A^{p}}) = 1$ for some $m$.
             So \[
                 \left[\cdots\left[\frac{G}{A^{p}}, \left[\frac{G}{A^{p}}, \frac{G}{A^{p}}\right]\cdots \right]\right] = \frac{A^p}{A^p}.  
             \]

             Now, we're going to make this group a little bit smaller by replacing $G$ with $A$
\[
                 \left[\cdots\left[\frac{G}{A^{p}}, \left[\frac{G}{A^{p}}, \frac{A}{A^{p}}\right]\cdots \right]\right] = \frac{A^p}{A^p}.  
             \]
             But $A^{p} = (p\Z)^{k}$. So $\frac{A}{A^{p}} \cong (\Z_p)^{k}$.
             Then $A_1 = [\frac{G}{A^{p}}, \frac{A}{A^{p}}]$ is a subgroup and it should be smaller because we should reach $1$.
             \[
                 \frac{A}{A^{p}} \cong \Z_p^{k} > A_1 > A_2 > \cdots > A_m = 1
             .\] 
             So $ A_1$ has at most $p^{k-1}$ elements. (This is where use that $p$ is prime)
             The maximum for $ A_2$ is $p^{k-2}$.
             And when $m =k$,  $A_m = 1$.
             The strong part is:  $k$ does not depend on $p$!

             So there exists an $m$ independent of $p$ such that
             \[
                 \underbrace{[G, \cdots [G, [G, A]\cdots]}_{m} \le  A^{p}
             .\] 
             
             But this is not really what we want,  we want this to end at a certain moment, and $A$ replaced by $G$.
             However, note that $\frac{G}{A}$ also satisfies the hypothesis of the theorem and $h(G / A) = h(G)$.
             So $\frac{G}{A}$ is nilpotent.
             So there is a $n \in \N_0$ such that $\gamma_n(G / A)  = \frac{A}{A}$. SO $\gamma_n(G) \le A$.
             But then we have that
             \[
                 \underbrace{[G, \cdots [G, [G, \gamma_m(G) ]\cdots]}_{m} \le  A^{p} \qquad \forall p
             .\] 
             So this is the $n + m$ term of the lower central seris.
             So 
              \[
                  \gamma_{m+n}(G) \le  \bigcap A^{p} = 1
             .\] 
             So this implies that $G$ is nilpotent.
    \end{itemize}
\end{proof}


