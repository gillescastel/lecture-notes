\lesson{8}{di 20 okt 2020 16:19}{}
\url{https://www.youtube.com/watch?v=1xR4AvJUL0Y}


\nsection{16}{2}{General properties of nilpotent groups}

\begin{lemma}[16.2.0, not in text]
    Let $G$ be a nilpotent group of class $\le c$.
    \begin{itemize}
        \item If $H\le G$, then $H$ is also nilpotent of class $\le c$.
        \item If $N \triangleleft G$ then $G / N$ is also nilpotent of class  $\le c$.
    \end{itemize}
\end{lemma}
\begin{proof}
    Let
    \[
    G_0 = 1 \le  G_1 \le  \cdots \le  G_c = G
    \] 
    be a central series.
    Claim: then  $H_i = G_i \cap H$ is a central series for $H$.
    We have to show
    \[
        [H, H_{i+1}] \le  H_i
    .\] 
    We have $H \le  H$ and $H_{i+1} \le G_i$, so we have that  $[H, H_{i+1}] \le H \cap G_i = H_i$.

    \hr

    For normal subgroup same way: $ (G_i N) / N$ wil be a central series for $G / N$.
\end{proof}

\begin{lemma}
    Let $G$ be a nilpotent group of class $ c\ge 2$.
    Thenany subgroup generated by $[G, G]$ and one element
    is nilpotent of class $<c$.
\end{lemma}
\begin{proof}
    Let $H = \left<a, [G, G] \right>$.

    We'll use the following fact
    \begin{itemize}
        \item If $K$ is a group such that  $\frac{K}{Z(K)}$ is cyclic, then $K = Z(K)$. 
            \begin{proof}
                Assume $K / Z(K) = \left<\overline{k} \right>$ with $k \in K$, so $\overline{k} = k Z(K)$.
                Then any element of $K$ is of the form  $zk^{l}$ with $z \in Z(K)$ and $l \in \Z$.
                But then if $z_i k^{\ell_i} \in K$ for $i = 1, 2$.
                Then
                 \[
                z_1 k^{\ell_1} z_2 k^{\ell_2} = 
                z_2 k^{\ell_1} k^{\ell_2} z_1 = z_2 k^{l_2} z_1 k^{l_1}
                ,\] 
                because $z_i$ is in the center and $k$ commutes with itself.
                This means that $K$ is abelian, so it is its own center.
            \end{proof}
    \end{itemize}

    Now, write out the scheme of problem 4 (TODO) for class $c$.
    Then we find that
    \[
        [G, G] \le  Z_{c-1}(G)
    .\] 
    But also $[G, G] \subset H$, so
    \[
        [G, G] \le Z_{c-1}(G) \cap H
    .\] 
    And by writing scheme for $H$ with $G_i \cap H$, we find that also 

    \[
        [G, G] \le Z_{c-1}(G) \cap H \le Z_{c-1}(H) \tag{$*$}
    .\] 

    Now, we have
    \[
        \frac{H}{Z_{c-1}(H)}
        \cong  \frac{H / Z_{c-2}(H)}{ Z_{c-1}(H) / Z_{c-2}(H) }
        \cong \frac{H / Z_{c-2}(H)}{ Z (H / Z_{c-2}(H)) } \tag{2}
    .\] 
    Here, we use the fact that $c \ge 2$ and also the definition of upper central series.
    Now, what is $H / Z_{c-1}(H)$? By $(*)$, we know that we've modded out at least  $[G, G]$, so as  $H = \left<a, [G, G] \right>$, we know that this group is cyclic.
    So $\frac{H}{Z_{c-1}(H)} = \left<\overline{a} \right>$ is cyclic.
    So we have
    \[
    \left<\overline{a} \right>
    \cong \frac{H / Z_{c-2}(H)}{ Z (H / Z_{c-2}(H)) }
    .\] 
    By the fact we've proved earlier, we find that $H / Z_{c-2}(H) = Z(H / Z_{c-2}(H))$, so this is the trivial group.

    Conclusion: $H = Z_{c-1}(H)$ by $(2)$.
    So  $H$ is nilpotent of class at most  $c-1$.
\end{proof}

Why is should we find this lemma interesting? 
This decreases the nilpotency class and a lot of proofs we'll discuss will be by induction (on class, length of series, \ldots).

\begin{theorem}[16.2.2]
    Let $G$ be a nilpotent group. 
    Then any subgroup $H \le G$ is subnormal in $G$.
\end{theorem}
\begin{definition}[Subnormal]
    $H$ is  subnormal in $G$ iff there exists a sequence $H_i \le G$ such that
    \[
    H = H_0 \le  H_1 \le  \cdots \le  H_n = G
    \] 
    and $H_i \triangleleft H_{i+1}$. So not normal in the total group, but normal in the next one!
\end{definition}
\begin{lemma}
    Normal subgroups are subnormal. Just take $ H_0 = H$, $ H_1 = G$.
\end{lemma}

Remember that any subgroup of an abelian group is normal.
This theorem is a generalisation.

Later, we will show that in the context of finite groups ``each subgroup is subnormal'' is equivalent to ``$G$ is nilpotent''.


\begin{proof}[of the theorem]
    Define $ H_0 = H$ and $H_{i+1} = N_G(H_i)$, where $N_G(H_i)$ is the normalizer of $H_i$, i.e. the largest subgroup of $G$ in which $H_i$ is normal. In other words:
     \[
         H_{i+1} = N_G(H_i) = \{g \in G  \mid  gH g ^{-1} = H\} 
    .\] 
    Claim: for any $i$, we have that  $Z_i(G) \le  H_i$.
    If we can prove this, we're done. 
    Indeed, for some index $Z_i(G)$ is the total group.

    By induction on  $i$, where  $i= 0$,  $Z_0(G) = 1 \le  H_0$.
    Assume that $Z_i \le H_i$.
    Let $z \in Z_{i+1}$ and $h \in H_i$.
    We want to show that $z \in H_{i+1}$.
    So we want to show that $z^{-1} h z \in H_{i}$
    \begin{align*}
        z^{-1} h z &= h (h^{-1} z^{-1} h z)\\
                   &= h [h, z]
    .\end{align*} 
    Then $[h, z] \in H_{i}$, because central $Z_i$ series.
    ($[h, z]$ is an element in the center modulo $z_i$.)
    So we get because $h \in H_i$, we have that $h [h, z] \in H_i$.
    This implies that $z \in N_G(H_i) = H_{i+1}$.
    So $Z_{i+1} \le H_{i+1}$.
\end{proof}

\begin{theorem}[16.2.7]
    If $G$ is a nilpotent group, then the set of torsion elements of  $G$, $\tau G$, forms a subgroup of  $G$.
    \[
    \tau G = \{g \in G  \mid  \exists  k \in \N_0 : g^{k} = 1\} 
    .\] 
    This does not hold in general!
\end{theorem}

If group is finite, then $\tau G = G$.
\begin{eg}
    $G = D_{\infty} = \left<a, b  \mid  ba  = a^{-1} b, b^2 = 1 \right>$.
    Elements are of the form $a^{z}$ and $a^{z}b$.
    Note: $(a^{z}b)^2 = a^{z} b a^{z} b = a^{0} b^2 = 1$.
    All elements of the form $a^{z}$ do not get to $1$ when taking powers.
    So torsion elements are
     \[
    \tau G = \{1, b, ab, a^{-1}b, a^2b, a^{-2}b, \ldots \} 
    .\] 
    This is not a group!  For example, $ab b = a \not\in  \tau G$.

    You can also write:
    \[
    D_{\infty} = \left\{ \begin{pmatrix}
        \pm 1 & z \\
        0 & 1
    \end{pmatrix}  \mid z \in \Z\right\} 
    .\] 
\end{eg}

So nilpotent groups are special!

\begin{proof}
    We use induction on the nilpotency class\footnote{All proofs start like this!}
    \begin{itemize}
        \item [Base]
            If $G$ is abelian, then this is rather easy. $|a| =|a^{-1}|$ and if $a^{l} = 1$ and $b^{k} = 1$ then $(ab)^{l k} = 1$.
        \item [Ind.] Assume that nilpotency class of $G $ is $c \ge 2$ and that the result holds for smaller nilpotency classes.
            Let $a \in \tau G$ and $b\in \tau G$.
            We need to show that $ab$ is also a torsion element.
            Consider the groups
            \[
                 A = \left<a, [G, G] \right> \qquad B = \left<b, [G, G] \right>
            .\] 
            Then both $A$ and  $B$ are nilpotent of class  $<c$. (Lemma)

            Note that $A \triangleleft G$, because $[A, G] \le  [G, G] \le  A$.
            Also $B \triangleleft G$ because of the same argument.
            Now by induction, $\tau A$  is a subgroup of $A$ which is moreover characteristic (This means: for all $\phi \in \operatorname{Aut}(A) : \phi( \tau A ) = \tau A$. This is obvious $\phi(a)^{n} = \phi(a^{n}) = \phi(1) = 1$)
            Hence, $\tau A \triangleleft G$. Indeed, let $g \in G$ then consider $\mu_g : A \to  A: a \mapsto g^{-1} a g$ is a group automorphism of $A$ (because $A \triangleleft G$ !)
            So $\mu_g(\tau A) = \tau A$ so  $\tau A \triangleleft G$.\footnote{This is a general argument: if you have a normal subgroup and you have a characteristic subgroup of your subgroup, then this subgroup has to be normal itself.}

            For similar reasons, $\tau B \triangleleft G$ in exactly the same way.

            Now, the product of two normal subgroups is again a subgroup:
            \[
                (\tau A )(\tau B) \triangleleft G
            .\] 
            Now, what about $a b ^{-1} \in \tau A \ \tau B$?
            Then 
            \[
                a b^{-1} \cdot (\tau B) \in \frac{\tau A \ \tau B}{\tau B}
            .\] 
            So $ab^{-1} \cdot \tau B = a \ \tau b$.
            Now, there exists some $k \in \N_0$ such that $a^{k}=1$,
            So  $(a \ \tau B)^{k} = (a b^{-1} \ \tau B)^{}\tau B$.
            So it follows that $(ab^{-1})^{k} \in \tau_B $.
            So there exists some $((a b^{-1})^{k})^{l} = 1$. So $ab^{-1}$ has finite order.
    \end{itemize}
\end{proof}
