\lesson{9}{di 27 okt 2020 10:44}{}

\begin{theorem}[16.2.3]
    Let $G$ be a nilpotent group and $H \triangleleft G$.
    If $H \neq 1$ then $H \cap Z(G) \neq 1$.
\end{theorem}

So every normal subgroup does intersect the center in a non-trivial way. So there are always non-trivial center elements in any normal subgroup.

\begin{proof}
    Let $G$ be nilpotent of class $c$.
    If $c = 1$, then abelian, so $G = Z(G)$, then the result is trivial.

    Assume $c>1$.
    \begin{description}
        \item [If $H \le Z(G)$], then we're done
        \item [Otherwise] Consider $ \frac{HZ(c)}{Z(G)} \triangleleft  \frac{G}{Z(G)}$.
            Then $H Z(G) / Z(G)$ is not trivial, and  $G / Z(G)$ is nilpotent of class  $c - 1$.\footnote{
                $Z(G / Z(G)) = Z_2(G) / Z_1(G)$ by definition of $ Z_2$ (`the second center').
                And moreover $ Z_2 (G / Z(G)) = Z_3(G) / Z_1(G)$ and so on, $Z_{c-1}(G / Z(G)) = \frac{Z_c(G)}{Z_1(G)} = G / Z(G)$.
                So if we mod out the center, we just kill one nilpotency class.
            }
            So we can apply the induction hypothesis.
            So
            \[
                \frac{HZ(G)}{Z(G)} \cap Z\left( \frac{G}{Z(G)} \right)  \neq 1
            .\] 
            But $Z(G / Z(G)) = Z_2(G) / Z(G)$, so 
            \[
                \frac{HZ(G)}{Z(G)} \cap  \frac{Z_2}{Z(G)} \neq 1
            ,\] 
            so there is non-trivial element in $HZ(G) \cap Z_2$ and $x \not\in Z(G)$.

            So $x = hz$ with  $h \in H$, $z \in Z(G)$ and also an element of $ Z_2(G) \setminus Z_1(G)$.
            But then it follows that $hz z^{-1} = h \in Z_2(G)$, because $Z(G)$ is a subgroup of  $ Z_2(G)$, but still $h \not\in Z\left( G \right) $.
            So 
            \[
                h \in Z_2(G) \setminus Z(G)
            .\] 
            But this is not what we want, we want an element in the center that is not trivial.
            But the previous statement translates to `there exists a $g \in G$ such that $[h,g] \neq  1$.'
            But $[h, g] \in H$\footnote{Why? $H$ is normal so  $[H, G] \le H$}
            Secondly because central series: $[Z_2(G), G] \le Z_1(G)$.
            So $[h, g] \in H \cap Z(G)$.
            This finishes the proof.
    \end{description}
\end{proof}

\begin{theorem}[16.2.5]
    Let $G$ be nilpotent.
    Assume that $A \le G$ such that $A [G, G] = G$.
    Then $A = G$.
\end{theorem}
\begin{proof}
    Assume that $A \neq G$.
    Define $A_i = A Z_i$ where  $Z_i = Z_i(G)$.
    Then $A_0 = A \neq G$.
    This is also an increasing sequence:
    \[
    G \neq A_0 \le  A_1 \le  A_2 \le  \cdots \le  A_c = G
    .\] 
    Let $n$ be such that $A_n \neq G$ and $A_{n+1} = G$.
    Note $A_i \triangleleft A_{i+1}$.\footnote {
    We need to show $A Z_i \triangleleft A Z_{i+1}$.
    Take conjugate of element an use fact that $a z_{i+1} = z_{i+1} a \mod Z_i$.
}
    Now, $\frac{A Z_{i+1}}{A Z_i}$ is abelian!
    Why? Consider the morphism 
    \[
    \frac{Z_{i+1}}{Z_i} \to  \frac{A Z_{i+1}}{A Z_i}: z Z_i \mapsto  z A Z_i
    .\] 
    This is well defined and also epi (onto), because we mod out the $A$ (check)
    The left group is abelian (quotient of central series), so the right one is as well. (Or show directly like in the footnote)
    So $\frac{A_{i+1}}{ A_i}$ are abelian.
    In particular, $\frac{A_{n+1}}{ A_n} = \frac{G}{A_n}$ is abelian. This implies that $[G, G] \le A_n$. Indeed
    \[
        \left[\frac{G}{A_n}, \frac{G}{A_n}\right] = \frac{A_n}{A_n} = 1
    ,\] 
    so $[G, G] \le A_n$.

    Now we're done!
    Why?
    $[G, G] \le A_n = A Z_n$.
    This implies that $[G, G] A \le  A_n$.
    But $[G, G] A= G$, so  $G \le A_n \le G$.
    This is contradiction. We assumed $A_n$ was not already $G$, but now we have that $G$ is a subgroup of  $A_n$.
\end{proof}

What's the use of this theorem?

\begin{eg}[Application, not in the text]
    Let $G$ be a nilpotent group.
    Let $\phi: G \to  G$ be a morphism.
    Then $\phi([G, G]) \le [G, G]$.\footnote{This is a general fact, not requiring nilpotency}
    So $\phi$ induces a well-defined morphism
    \[
        \overline{\phi}: \frac{G}{[G, G]} \to  \frac{G}{[G, G]}
    .\] 
    Claim: $\phi$ is surjective iff $\overline{\phi}$ is surjective.
\end{eg}
\begin{explanation}
    If $\phi$ is surjective then certainly $\overline{\phi}$ is as well.
    Conversely, suppose $\overline{\phi}$ is surjective.
    So then $\operatorname{Im}(\overline{\phi}) = \frac{G}{[G, G]}$.
    This implies that $\frac{\operatorname{Im}(\phi) [G, G]}{[G, G]} = \frac{G}{[G, G]}$, which implies that
    \[
        \operatorname{Im} (\phi) [G, G] = G
    .\] 
    By the previous result, $\operatorname{Im}(\phi) = G$.
\end{explanation}

Strange result:

\begin{theorem}[16.2.6]
    In a nilpotent group $G$, every maximal normal abelian subgroup\footnote{If $A \le  B \triangleleft G$ and $B$ is abelian, then $A = B$}  $A$ is its own centralizer.
\end{theorem}
\begin{proof}
    Let $H = C_G(A) = \{g \in G  \mid  g a = a g \quad \forall a \in A\} $.
    Note $H \triangleleft G$. (exercise).
    Claim: for all $i = 0, 1, \ldots$ we have
    \[
    H \cap Z_i \le A
    .\] 
    At a certain moment, we have $Z_i = G$ so then  $H \le A$.
    \begin{itemize}
        \item[$i=0$] We are done: $ Z_0 = 1$.
        \item[$i>0$] Assume $H \cap Z_i \le A$ and let $x \in H \cap Z_{i+1}$.
            Let $g \in G$, then because $H$ is normal,  $[H, G] \le H$ and $[Z_{i+1}, G] \le Z_i$, we have
            \[
                [x, g]  \in H \cap Z_i \le A
            .\] 
            This implies that $\left<a, A \right> \triangleleft G$.\footnote{Check this is normal for yourself! $g^{-1} x^{-1} g x \in A$, so $g^{-1} x^{-1} g \in \left<x, A \right>$. Now do the rest\ldots}
            Moreover, this group is abelian. Indeed, $A$ is ablian and $x \in H$ and $H$ is the centralizer of  $A$. 
            This is an abelian normal subgroup of $G$, so  $\left<x, A \right> = A$ by maximality of $A$. So $x \in A$.
    \end{itemize}
\end{proof}

\begin{corollary}
    Under conditions of the previous theorem, there is an injective homomorphism
    \[
        \phi: G / A \to  \operatorname{Aut}(A)
    .\] 
\end{corollary}
\begin{proof}
    Let \[
        \psi: G \to  \operatorname{Aut}(A): g \mapsto (a \mapsto g a g^{-1})
    \]
    We can do this for any normal subgroup.
    Now, \begin{align*}\Ker \psi &= \{ g \in G  \mid \psi(g) = 1_A\}\\
        &= \{ g \in G  \mid  g a ^{-1} g = a \quad \forall  a \in A\} \\
        &= C_G(A) = A.
    \end{align*}
    So modding out by the kernel, we have something injective:
    \[
        \psi: \frac{G}{A} \to \operatorname{Aut}(A)
    .\] 
\end{proof}
\begin{eg}
    If $A = \Z^{k}$, then $\operatorname{Aut}(A)$ are matrices, which gives then a nice representation of the group $G / A$.
\end{eg}



Recall: if $G$ is nilpotent group, then  $\tau G \triangleleft  G$. Then $G / \tau G$ is a torsion free group!

\begin{theorem}
    In a torsion-free nilpotent group, taking roots is a well-defined operation.
    In other words:
    \[
    \forall a, b \in G: \forall n \in \N_0: a^{n} = b^{n} \implies a = b
    .\] 
\end{theorem}
\begin{proof}
    By induction of the nilpotency class of $G$.
    If $G$ is abelian, then $a^{n} = b^{n}$ implies that $a^{n} b^{-n}=1$, so $(a b^{-1})^{n} = 1$. Since the group is torsion free, $ab^{-1} = 1$ so $a = b$.

    Now assume $G$ is nilpotent of class  $c\ge 2$
    Then $\left<a, [G, G] \right>$ is nilpotent of class $<c$.
    We know that $\left<a, [G, G] \right> \triangleleft G$.\footnote{Why? Any group containing the commutator subgroup is normal! $[G, G] \le H$ so $[H, G] \le H$.} So $b a b^{-1} \in \left<a, [G, G] \right>$.
    So then
    \[
        (b a b^{-1})^{n} = b a^{n} b^{-1} = b b^{n} b^{-1} = b^{n} = a^{n}
    .\] 
    So because we're in a lower nilpotency class, by induction then $b a b^{-1} = a$.
    This shows that $ba = ab$.
    So $a$ and  $b$ commute. So we can do the same argument as in the beginning!!
    \[
        a^{n} = b^{n} \implies a^{n} b^{-n} = 1 \implies (a b^{-1})^{n} = 1 \implies ab^{-1} = 1 \implies a = b
    .\] 
\end{proof}

Interesting corollary: Torsion free nilpotent groups. Then all quotients of upper central series are torsion free too! This does not hold for lower central series.

\begin{theorem}[16.2.12, Fitting]
    If $A \triangleleft G$ and $B \triangleleft G$. If $A$ is nilpotent of class $s$ and $B$ is nilpotent of class $t$.
    Then $AB \triangleleft G$ is also a nilpotent group of class  $\le s + t$.
\end{theorem}
\begin{remark}
    For all $a, b, c \in G$ we have
    \[
        [ab, c] = [a,c]^{b} [b,c] = [a^{b}, c^{b}] [b, c]
    ,\] 
    where $x^{b} = b^{-1} x b$.
\end{remark}
\begin{proof}
    Let $a \in A$, $ b \in B$ and $c \in N\triangleleft G$, where $N$ is a normal subgroup of  $G$.
    Then
    \[
        [ab, c] = [b^{-1}ab, b^{-1} c b][b, c]
    .\] 
    Now, $b^{-1} a b =: a'\in A$ and $b^{-1} c b=: c' \in N$.
    So
    \[
    [ab, c] \in  [A, N][B, N]
    .\] 
    Claim $\gamma_i(AB) \le  \prod [H_1, [H_2, [H_3, \ldots, H_i]]\cdots]$ where $H_j$ is  $A$ or  $B$.
    So more concretely, 
    \begin{align*}
        \gamma_2(AB) = [AB, AB] &\le  [A, AB] [B, AB]\\
                                &\le  [A, A] [A, B] [ B,A] [B, B]
    .\end{align*} 
    Then
    \[
        \gamma_{i+1}(AB) = [AB, \gamma_i(AB)] = \ldots
    \] 
    This is all what we needed to show!
    Indeed,
    \[
        \gamma_{s+t+1}(AB) \le  \prod [H_1, [H_2, [\cdots, [H_{s+t}, H_{s+t+1}]] ]
    .\] 
    We claim that all of these terms are trivial.
    For example, take $s = 3$ and  $t = 4$.
    Let's take an element in the product, say
     \[
         [A, [B, [B, [A, [B, [A, [A, A]]]]]]]
    .\] 
    We can forget all the $B$'s because  $[B, N] \le  N$ for any normal subgroup. We start with $[A, A]$, which is a normal subgroup. Taking commutators,  $[N, M] \triangleleft G$ is still normal.
    Forgetting about $B$ makes this group only bigger. 
    We get
    \[
        [A, [B, [B, [A, [B, [A, [A, A]]]]]]] \le \gamma_5(A) = 1
    .\] 

    So in the product in each term, or there's at least $s+1$  $A$'s or  $t+1$  $B$'s, so then  $\le  \gamma_{t+1}(B) = 1$.
    This completes the proof.
\end{proof}
