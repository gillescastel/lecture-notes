\lesson{2}{di 29 sep 2020 12:45}{Introduction}

We also have the following important result (not requiring irreducible representations.)

\begin{theorem}[2.13, Maschke's Theorem]
    Let $k$ be any field, and let $\rho: G \to  \GL_k(V)$ be a representation.
    Let $W$ a $G$-subspace of $V$.
    Then there exists $W'$, a $G$-subspace of $V$ such that $V = W \oplus W'$.
\end{theorem}

In terms of matrices, this means, that we have the following
\[
\begin{pmatrix}
    \rho|_W & 0 \\
    0 & \rho|_{W'}
\end{pmatrix}
.\] 

\begin{proof}
    We will show that there is a projection $ p_0: V \to  W$ which is a $G$-map.
    This is enough. Why?
    $ p_0:V \to  W$ is a projection. So $ p_0(V) \subset W$, $ p_0|_W = 1_{W}$ and $ p_0^2 = p_0$.
    Then $V = \Ker p_0 \oplus \im p_0$. (Check for yourself)
    Take $W' = \Ker p_0$.


    Take $p: V \to  W$ a projection (that is not necessarily a $G$-map.)
    Define
    \[
        p_0: V \to  V : v\mapsto \frac{1}{|G|} \sum_{h \in G} h\cdot p(h^{-1}\cdot v)
    .\] 
    \begin{itemize}
        \item The image is certainly $W$, because $W$ is a $G$-invariant space ($p(\ldots)$ belongs to $W$, so does $h\cdot p(\ldots)$).
        \item For $v = w \in W$. We want to prove that $ p_0(v) = v$. We get
            \begin{align*}
                p_0(w) &=\frac{1}{|G|} \sum _{h \in G} h \cdot p(h^{-1} \cdot w)\\
                &= 
                \frac{1}{|G|} \sum _{h \in G} h \cdot h^{-1} \cdot w \\
                                    &= \frac{1}{|G|} \sum _{h \in G} w \\
                                    &= w \\
            .\end{align*}
    \end{itemize}

    So $\im (p_0) = W$ and $p_0$ is also a projection onto $W$.
    We now claim that $p_0$ is a $G$-map
    \begin{align*}
        p_0(g\cdot v)
        &= \frac{1}{|G|} \sum_{h \in G} h\cdot p(h^{-1}\cdot g \cdot v)\\
        &= \frac{1}{|G|} \sum_{h \in G} g\cdot  g^{-1}\cdot h\cdot p(h^{-1}\cdot g \cdot v)\\
        &= g\cdot \left(\frac{1}{|G|} \sum_{h \in G} g^{-1}\cdot h\cdot p(h^{-1}\cdot g \cdot v)\right)\\
        &= g\cdot \left(\frac{1}{|G|} \sum_{h \in G} g^{-1}h\cdot p(h^{-1}g \cdot v)\right)\\
        &= g\cdot p_0(v)
    .\end{align*}

    This proves that $p_0$ is a $G$-map.
    Take $W' = \Ker p_0$.
\end{proof}

\begin{remark}
    Also works for compact topological groups.
    Sums become integrals over Haar measure, $|G|$ becomes volume of $G$, \ldots
\end{remark}

% TODO: subscripts escape from math mode doesn't work with tab \ldots


An important corollary is that any representation can be described using irreducible representation.

\begin{theorem}[2.14]
    Let $\rho: G \to  \GL_k(V)$ be a representation.
    Then there exists subspaces $U_1, U_2, \ldots, U_k$ of $V$ such that each $U_i$ is a  $G$-subspace and $\rho|_{U_i}$ is irreducible and $V = U_1 \oplus U_1 \oplus \cdots \oplus U_k$.
\end{theorem}

As a matrix:

\[
\begin{pmatrix}
    \rho|_{U_1} & 0 &0 &0\\
    0 & \rho|_{U_2} & 0&0\\
    0 & 0& \ddots & 0\\
    0 & 0 &0&\rho|_{U_3} 
\end{pmatrix}
.\] 

\begin{proof}
    Proof is induction on the dimension of $V$.

    \begin{description}
        \item[$\dim V = 1$] We are done. The representation is irreducible, $V = U_1$.
        \item [$\dim V > 1$] Either $V$ is irreducible (then $V = U_1$)
            If not, then there exists $W$, a $G$-space of $V$ which is not trivial.
            Then, by Maschke's Theorem, there exists a non-trivial $W'$ such that  $V = W \oplus W'$ and $W'$ is also a  $G$-subspace.
            This implies that $0 < \dim W', \dim W < \dim V$.
            Then by induction  $W = U_1 \oplus \cdots \oplus U_k$, and same for $W'$.
            Then $V = U_1 \oplus \cdots \oplus U_k \oplus \cdots \oplus U_n$
    \end{description}
\end{proof}


This decomposition is unique in the following sense.
The spaces might not be the same, but the representation look alike.
We can pair the spaces with same dimension. The actions are $G$-equivalent.

\begin{eg}
    Take $\rho: G \to \GL(\R^2): g \mapsto (v \mapsto v)$.
    Then $R = \left< v\right> \oplus \left<w \right>$ for any $v, w$ independent.
\end{eg}

We will prove the uniqueness later.


\begin{definition}
    Let $\rho: G \to  \GL_k(V)$ be a representation.
    Then \[
    V^{G} = \{v \in V: g\cdot v = v \text{ for all $g \in G$ }\} 
\]
\end{definition}

\begin{prop}[2.15]
    Let \[
        \epsilon: V \to  V: v \mapsto \frac{1}{|G|} \sum_{g \in G} g\cdot v.
    \]\footnote{Dividing by  $|G|$ uses that  $G$ is finite, but also that char = 0.}

    \begin{enumerate}[(a)]
        \item $g\cdot  \epsilon(v) = \epsilon(v)$ for all $g, v$, which is to say that $\im \epsilon \subset V^{G}$
        \item $\epsilon$ is $G$-linear
        \item For all $v \in V^{G}: \epsilon(v) = v$, which is to say that $\epsilon|_{V^{G}} = 1_{V^{G}}$
    \end{enumerate}
    These properties imply that $\epsilon: V \to  V^{G}$ is a $G$-projection onto $V^{G}$.
    This implies that $V = \Ker \epsilon \oplus V^{G}$
    Choosing a basis $\mathbf{v} = \{v_1, \ldots, v_k, v_{k+1}, \ldots, v_n\}$, such that the last vectors form a basis of $V^{G}$, we find that 
    $[\epsilon]_{\mathbf{v}} = \begin{pmatrix}
        0 & 0 \\
        0 & I
    \end{pmatrix}$
\end{prop}

\begin{remark}
    Note that this is $\epsilon_+$ from some time ago, where  $G = \Z_2$.
\end{remark}
\begin{proof}
    Trivial.
\end{proof}

\nsection{2}{3}{New representations from old ones}

Some examples:

\begin{itemize}
    \item Let $\rho, \sigma$ be representations of  $G$ over $V, W$.
        Take $U = V \oplus W$.
        Then \[\rho \oplus \sigma: G \to  \GL_k(V \oplus W) : g\mapsto v + w \mapsto  \rho(g) v + \sigma(g) w.\]
    \item Consider $V \otimes W$.
        Then  \[\rho\otimes \sigma: G \to  \GL(V \otimes W): g \mapsto v \mapsto (\rho(g) v) \otimes (\sigma(g) w)\]
    \item Consider $\Hom_k(V, W)$, the vector space of linear maps from  $V$ to $W$.
        \begin{align*}
            \psi: G &\longrightarrow \GL_k(\Hom_k(V, W))\\
            g &\longmapsto (f \mapsto (v \mapsto \sigma(g) f( \rho(g)^{-1} v )))
        .\end{align*} 
\end{itemize}

\begin{ex}
    Check that this is a representation and this
\end{ex}


% TODO: oplus becomes operator
