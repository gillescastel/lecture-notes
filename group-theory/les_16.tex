\lesson{16}{di 17 nov 2020 12:45}{}
\url{https://youtu.be/c1haPsdBdmA}
\paragraph{Cochain complex}
\begin{remark}
    Def 1.29 is wrong in the text, indices not at the right place.
\end{remark}
\[
\cdots \to  C^{n-1} \xrightarrow{d^{n-1}}  C^{n} \xrightarrow{d^{n}}   C^{n+1} \to  \cdots
\] 
with again $d^2 = 0$.
We have coboundaries and cocycles.
\[
    B^{n}(C^*) = \im \subset  \Ker = Z^{n}(C^{*})
.\] 
Etc \ldots
We can also view a cochain complex as a chain complex just by renumbering.

If $0 \to  A^{*} \to  B^{*} \to  C^{*} \to 0$ is a short exact sequence of cochain complexes, we obtain a long exact sequence in cohomology
\[
H^{n}(A^{*}) \to  H^{n}(B) \to  H^{n}(C) \xrightarrow{} H^{n+1} (A) \to   \cdots
.\] 

\nchapter{2}{Tor and Ext}
\nsection{2}{1}{Derived functors}

\begin{definition}
    Let $A$ be an $R$-module. Then a projective resolution of $A$ is an \emph{exact} sequence of $R$-modules $P_*$:
\[
\cdots \to P_3 \xrightarrow{d_3} P_2 \xrightarrow{d_2} P_1 \xrightarrow{d_1} P_0   \xrightarrow{\epsilon} A \to  0
,\] 
where every $P_i$ is a projective module. 

\end{definition}

\begin{definition}
    The deleted projective resolution of $A$, $(P_A)_*$ is the sequence 
    \[
        \cdots \to P_3 \xrightarrow{d_3} P_2 \xrightarrow{d_2} P_1 \xrightarrow{d_1} P_0  \to  0
    .\] 
    This is not exact per se, but we can consider it as a chain complex.\footnote{Just add zeros at the end}
    
    Then we have
    \[
        H_n(P_A) = \begin{cases}
            0 &\text{if $n = 0$}\\
            A & \text{if $n \neq 0$}.
        \end{cases}
    \] 
\end{definition}

\begin{theorem}
    Every $R$-module has a projective (even free) resolution.
\end{theorem}
\begin{proof}
    Every module $A$ is a quotient of a free module, so we have
    \[
        P_0 \xrightarrow{\epsilon} A \to  0
    .\] 
    Consider $K_1 = \Ker \epsilon$. Then this can also be viewed as a quotient of a free module \ldots
    \[
        \begin{tikzcd}
            P_1 \arrow[rr, "d_1"] \arrow[dr, "\epsilon"]& &P_0 \arrow[r, "\epsilon"] & A \arrow[r, ""] & 0\\
                                                       & K_1 \arrow[ur, "", hook]& &  
        \end{tikzcd}
    \]
    This way we can construct a projective resolution.
\end{proof}


\begin{theorem}
    Let $A, B$ be two $R$-modules. Let $f: A \to  B$ be an $R$-module map.
    Consider the diagram
    \[
        \begin{tikzcd}
            P_3 \arrow[r, ""] \arrow[d, "f_3"]& 
            P_2 \arrow[r, ""] \arrow[d, "f_2"]& 
            P_1 \arrow[r, ""] \arrow[d, "f_1"]& 
            P_0 \arrow[r, "\epsilon"] \arrow[d, "f_0"]& 
            A \arrow[r, ""] \arrow[d, "f"]& 
            0\\
            P_3' \arrow[r, ""] & 
            P_2' \arrow[r, ""] & 
            P_1' \arrow[r, ""] & 
            P_0' \arrow[r, "\epsilon'"] & 
            B \arrow[r, ""] & 
            0\\
        \end{tikzcd}
    \]
    If the top sequence is a complex and $P_i$ is projective, and the bottom sequence is exact, \footnote{So if we have projective resolutions both conditions are satisfied.}
    Then we can define maps $f_i: P_i \to  P_i'$ such that all squares in the diagram are commutative.
    So the total map $\tilde{f}$ is a chain map.

    Moreover, this $ \tilde{f}$ is unique up to homotopy!
\end{theorem}
\begin{proof}
    We prove only the existence. Read uniqueness in the text.

    Construction of $f_n$ by induction.
    Start with $f_0$, 
    \[
        \begin{tikzcd}
            & P_0  \arrow[d, "\epsilon"] \arrow[ddl, "f_0", dashed]&\\
            & A \arrow[d, "f"]& \\
            P_0' \arrow[r, "\epsilon'"]& B \arrow[r, ""]&0 \\
        \end{tikzcd}
    \]
    Now, without the primes is projective, bottom row is exact, so by definition of projective module, we have the map $f_0$. This is not unique!
    Assume $f_0, \ldots, f_{n-1}$ have been constructed.
    \[
        \begin{tikzcd}
            P_n \arrow[r, "d_n"] &
            P_{n-1} \arrow[r, "d_{n-1}"] \arrow[d, "f_{n-1}"]&
            P_{n-2} \arrow[d, "f_{n-2}"]\\
            P'_n \arrow[r, "d'_n"] &
            P'_{n-1} \arrow[r, "d'_{n-1}"] &
            P'_{n-2}\\
        \end{tikzcd}
    \]
    Let us rewrite\ldots
    \[
        \begin{tikzcd}
            & P_n \arrow[d, "d_n"]& \\
            & P_{n-1} \arrow[d, "f_{n-1}"]& \\
            P_n' \arrow[r, "d_n'"] & ? \arrow[r, ""]& .
        \end{tikzcd}
    \]
    Because it should be surjective, $?$ should be $\Im d_n'$. But that is not the image of  $P_{n-1}$ under $f_{n-1}$, so doesn't work. WRONG:
    \[
        \begin{tikzcd}
            & P_n \arrow[d, "d_n"]& \\
            & P_{n-1} \arrow[d, "f_{n-1}"]& \\
            P_n' \arrow[r, "d_n'"] & \im d_n' \arrow[r, ""]& 0.
        \end{tikzcd}
    \]
    To fix this: consider
    \[
        \begin{tikzcd}
            & P_n \arrow[d, "d_n"]& \\
            &  \arrow[d, "f_{n-1}"]& \\
            P_n' \arrow[r, "d_n'"] & \im d_n' \arrow[r, ""]& 0.
        \end{tikzcd}
    \]
    So prove: $f_{n-1}  \circ d_n(p) \in \im d_n'$ for all $p \in P_n$.
    But $\im d_n' = \Ker d_{n-1}'$.
    So enough to prove that $d_{n-1}' ( f_{n-1}(d_n(p))) = 0$.
    By commutativity, we have that this is
    \[
        f_{n-2}  \circ  d_{n-1}  \circ  d_n p = f_{n-2}(0) = 0
    .\] 
\end{proof}

\begin{definition}[Additive functor]
    Let $A, B$ be two $R$-modules. and $f,g: A \to  B$ be two $R$-module homomorphisms.
    Then $T$ is an additive functor if
    \[
        T(f+g) = T(f) + T(g)
    .\] 
\end{definition}
\begin{eg}
    $\Hom(X, -)$ is an additive functor from cat of $R$-modules to cat of abelian groups.
    Another example $* \otimes -$.
    Other side works to.
\end{eg}

Fix an additive functor $R$ from $R$-mod to $S$-mod.
For any $R$-mod $A$, we also fix a projective resolution $P_A$.\footnote{Later we will show it doesn't really depend on $P_A$}
Consider the deleted one and apply $T$ :

\[
    \begin{tikzcd}
        T(P_3) \arrow[r, "T(d_3)"] & 
        T(P_2) \arrow[r, "T(d_2)"] & 
        T(P_1) \arrow[r, "T(d_1)"] & 
        T(P_0) \arrow[r, ""] & 
        0
    \end{tikzcd}
\]
Note that $T(0) = 0$ because we have an additive functor (abelian group morhpism).
Also $T(d_{n-1})  \circ  T(d_n) = 0$.
This means that the resulting sequence is again a chain complex.

Then

\begin{definition}
    \[
        (L_n T)(A) = H_n(T(P_*))
    .\] 
\end{definition}

Let $f: A \to  B$ be an $R$-map. Then we have maps $f_n$ by a previous theorem:


\[
    \begin{tikzcd}
        P_2 \arrow[r, "d_2"] \arrow[d, "f_2"]&
        P_1 \arrow[r, "d_1"] \arrow[d, "f_1"]&
        P_0 \arrow[r, "\epsilon"] \arrow[d, "f_0"]&
        A \arrow[r, ""] \arrow[d, "f"]&
        0\\
        P_2' \arrow[r, "d_2'"]  & 
        P_1' \arrow[r, "d_1'"]  & 
        P_0' \arrow[r, "\epsilon'"] &
        B \arrow[r, ""] &
        0\\
    \end{tikzcd}
\]

Now delete $A$ and apply $T$

\[
    \begin{tikzcd}
        T(P_2) \arrow[r, "T(d_2)"] \arrow[d, "T(f_2)"]&
        T(P_1) \arrow[r, "T(d_1)"] \arrow[d, "T(f_1)"]&
        T(P_0) \arrow[r, ""] \arrow[d, "T(f_0)"]&
        0\\
        T(P_2') \arrow[r, "(d_2)'"]  & 
        T(P_1') \arrow[r, "(d_1)'"]  & 
        T(P_0') \arrow[r, ""] &
        0\\
    \end{tikzcd}
\]
This diagram is also commutative, so this is a chain map.
So we have a $T(f_N)_*: H_n(T(P_*)) \to  H_n(T(P_*'))$.
If we choose different $f$'s, then they will be chain homotopic, so the $T(f_n)$ will be as well, so they will induce the same map on homology level.

So $(L_n T) (f)$ is uniquely determined.
Checking composition, identity, \ldots we see hat $(L_n T)$ is a functor for all  $n$.

The only problem right now is that we have fixed one projective resolution for a module.
What happens if we pick another?

Suppose $P_A$ and  $\tilde{P}_A$ are  different resolutions. Then we have functors
$ (LT)_n A$  and $\tilde{LT}_n (A)$.
Fact: the functors $LT$ and  $\tilde{LT}$ are naturally equivalent.
This means that for any $A$ there is an isomorphism

 \[
     \eta_A: (LT)_n (A) \to  (\tilde{LT})_n(A)
\] 
so that for any $R$-map $f: A \to  B$ the following diagram commutes
\[
    \begin{tikzcd}
        LT_n(A) \arrow[d, "LT_n(f)"] \arrow[r, "\eta_A"] & \tilde{LT}_n(A) \arrow[d, "\tilde{LT}_n(f)"]\\
        LT_n(B) \arrow[r, "\eta_A"] & \tilde{LT}_n(B)
    \end{tikzcd}
\]

\begin{proof}
    Sketch of the proof: see text.
    Choose two projective resolutions.
    Pick map $A \to  A$ as the identity, and we get maps between $P_i \to  \tilde{P}_i$.
    Apply $T$.
    Forget about $A$.
    Then consider $LT_n(A) \xrightarrow{(f_n)_*} \tilde{LT}_n(A)$, induced map.
    We define this is $\eta_A$.
    This is independent because  $(f_n)_*$ unique up to homotopy.
    Claim: this is an isomorphism.
    Why? We can do the same in the other direction.
    Composing the two diagrams, we have a bunch of identity maps, so the induced map of composition is identity.
\end{proof}
