\lesson{20}{di 01 dec 2020 12:53}{}

We will express first cohomology in terms of (principal) derivations.

Relation beten derivations and $G$-module maps.

Consider
\begin{align*}
    \Hom_{\Z G}(IG, A) &\longrightarrow \Der(G,A) \\
    \phi &\longmapsto D_\phi = (g \mapsto \phi(g-1))
.\end{align*}

Let us first check that $D_\phi$ is a derivation.

 \begin{align*}
     D_\phi(gh) &= \phi(gh-1)\\
            &= \phi(gh-g + g - 1)\\
            &= \phi(gh - g)  + \phi(g-1)\\
            &= g \phi(h-1) + \phi(g-1)\\
            &= g D_\phi(h) + D_\phi(g)
.\end{align*} 

This is a morphism of abelian groups!

Even more! This is an isomorphism!

\begin{itemize}
    \item Injective.
        Suppose $D_p: G \to  A$ is the zero map.
        Then $D_\phi(g) = 0$ for all  $g \in G$.
        So $\phi(g-1) = 0$ for all  $g \in G$.
These elements form a basis $IG$ as a free abelian group.
So  $\phi(x) = 0$ for all  $x \in IG$.

    \item Surjective.
        Let $D \in \Der(G, A)$.
        Define the morphism of abelian groups $\phi: IG \to  A$ by letting
        \[
            \phi(g-1) = D(g)
        .\] 
        We can do this because $IG$ is the free abelian group on the elements $g-1$.

        To show that $\phi$ is a $G$-map, it suffices to show that $\phi(g(h-1)) = g\phi(h-1)$.
        (if we can do for generators, we can do it for all elements).
         \begin{align*}
             \phi(g(h-1)) &= \phi(gh - g)\\
                          &= \phi(gh-1 - (g-1))\\
                          &= D(gh) - D(g)\\
                          \shortintertext{Because derivation:}
                          &= gD(h) + D(g) - D(g)\\
                          &= g D(h)\\
                          &= g \phi(h-1)
        .\end{align*}
\end{itemize}

\begin{theorem}[6.4.5]
    $H^{1}(G,A) \cong \frac{\Der(G, A)}{\PDer(G, A)}$.
\end{theorem}
\begin{proof}
    Consider the ses of $G$-modules
    \[
    0 \to  IG \to  \Z G \to  \Z \to  0
    .\] 
    Then this leads to a long exact sequence of $\Ext ^{n}_{\Z G} (-, A)$-functors.
    \[
        0 \to \Hom_{\Z G}(\Z, A) \to  \Hom_{\Z G}(\Z G, A) \to  \Hom_{\Z G}(IG, A) \to \Ext ^{1}_{\Z G}(\Z, A) \to  \Ext ^{1}_{\Z G}(\Z G,A) \to  \cdots
    .\] 
    For ext functors, they are zero if first slot is projective module:
    \[
        0 \to \Hom_{\Z G}(\Z, A) \to  \Hom_{\Z G}(\Z G, A) \to  \Hom_{\Z G}(IG, A) \to \Ext ^{1}_{\Z G}(\Z, A) \to  0 \to  \cdots
    .\] 
    $\Ext_{\Z G}^{1}(\Z, A)$ is the object we're looking for, i.e. $H^{1}(G, A)$.
    \[
        0 \to \Hom_{\Z G}(\Z, A) \to  \Hom_{\Z G}(\Z G, A) \to  \Hom_{\Z G}(IG, A) \to H^{1}(G, A) \to  0 \to  \cdots
    .\] 
    We know that $\Hom_{\Z G}(IG, A) \cong \Der(G, A)$;
    \[
        0 \to \Hom_{\Z G}(\Z, A) \to  \Hom_{\Z G}(\Z G, A) \to  \Der(G, A) \to H^{1}(G, A) \to  0 \to  \cdots
    .\] 

    We also know that $\Hom_{\Z G}(\Z G, A) \cong A$:

    \[
        0 \to \Hom_{\Z G}(\Z, A) \to  A \to  \Der(G, A) \to H^{1}(G, A) \to  0 \to  \cdots
    .\] 
    We also know that $\Hom_{\Z G}(\Z, A) = A^{G}$, so we get
    \[
        0 \to A^{G} \to  A \to  \Der(G, A) \to H^{1}(G, A) \to  0 \to  \cdots
    .\] 
    So we have to know that the image of $A$ is inside  $\Der(G, A)$, and then we know that  $H^{1}(G,A)\cong \frac{\Der(G, A)}{\ldots}$.

    So pick element $a \in  A$. This corresponds to $f \in \Hom_{\Z G}(\Z G, A)$ with  $f(1) = a$.
    Remember that $IG \to  \Z G$ was the inclusion map, so the arrow in other way is restriction.
    So $f$ becomes  $f|_{IG}$.
    This corresponds to a derivation.
    It is given by
    \[
        D(g) = f(g-1) = f(g) - f(1) = g f(1) - f(1) = ga - a = D_a(g)
    .\] 

    Conclusion:
    \[
        H^{1}(G, A) \cong \frac{\Der(G, A)}{\PDer(G, A)}
    .\] 
\end{proof}


\begin{eg}
    Suppose $A$ is a trivial  $G$-module.
    \begin{itemize}
        \item Then $D \in \Der(G, A)$ iff $D(gh) = gD(h) + D(g) = D(h) + D(h)$.
            So  $D \in \Hom_{\text{group}}(G, A)$.
        \item $\PDer(G, A) \ni ga -a = 0$, so  $\PDer = 0$.
    \end{itemize}
    So we have
    \[
        H^{1}(G, A) = \frac{\Hom_\text{group}(G, A) }{1} = \Hom_{\text{group}}(G, A)
    .\] 
\end{eg}

\subsection*{Cohomology of a free group}
We already know $H^{0}$ and $H^{1}$.
We will prove that $H^{>1} = 0$.

\begin{definition}
Let $M$ be a  $G$-module, then we can define a group $M \rtimes G$ where as a set 
\[
M \rtimes  G = M \times G
,\] 
and the product is 
\[
    (m_1, g_1) (m_2,g_2) = (m_1 + g_1m_2, g_1g_2)
.\] 
\end{definition}
\begin{eg}
    Let $A$ be any abelian group.
    Then  $A$ is a  $G = \Aut(A)$-module.
    $A \rtimes \Aut(A)$.
\end{eg}

\begin{lemma}
$M$ is a normal subgroup of  $M \rtimes G$ and  $\frac{M \rtimes G}{M} \cong G$.
\end{lemma}

\begin{definition}
A section is a set map $s : G \to  M \rtimes G$ such that $ p  \circ  s  = 1_G ,$ where $p(m,g) = g$. 
\end{definition}
So $s: G \to  M \rtimes G$ is determined by map $d: G \to  M$: $s(g) = (d(g),g)$. 


Claim:
\begin{prop}
    The section $s$ (determined by  $d$ by the map  $d: G \to  M$ ) is a morphism of groups iff $d \in \Der(G, M)$
\end{prop}

\begin{remark}
    When $s$ is a morphism, it is called a \emph{splitting}
\end{remark}
\begin{proof}
    Computation.
    $s: G \to  M \rtimes G$ determined by $d$ is a morphism.
    iff 
     \[
         s(gh) = s(g) s(h) \quad \forall g, h \in G
    \] 
    iff
    \[
        (d(gh),gh) = (d(g),g)(d(h),h)
    \] 
    iff
    \[
        (d(gh), gh) = (d(g) + g d(h), gh)
    \] 
    iff
    \[
    d \text{ is a derivation}
    .\] 
\end{proof}
\begin{lemma}[6.4.8]
    If $G$ is a free group on the set $X$, then $IG$ is a \emph{free}  $\Z G$-module on the set $X - 1$,
    \[
    X-1 = \{x -1  \mid  x \in X\} 
    .\] 
\end{lemma}
\begin{proof}
    We already know that it generates, but now we show that its freely generated!
    We need to show that the universal property for free modules holds.
    Let $A$ be any  $\Z G$-module and $\phi: X -1 \to  A$ any map.
    Given this, we need to show existence of dashed arrow.
    \[
        \begin{tikzcd}
            X-1 \arrow[dr, "\phi"] \arrow[r, "", hook]& IG \arrow[d, "\tilde{\phi}", dashed]\\
                                   & A
        \end{tikzcd}
    \]
    If there exists one, it is determined.
    At most one $\tilde{\phi}$, because $\phi$ is determined on generators \ldots

    We show that there exists a $\tilde{\phi} \in  \Hom_{\Z G}(\Z G, A)$ making the diagram commute.

    Remember, $\Hom_{\Z G}(IG, A) \cong \Der(G, A)$, where $\tilde{\phi}\mapsto D_{\tilde{\phi}}$.

    So it is enough to show that the corresponding derivation exists.
    We use the fact that $G$ is a free  group to construct the derivation.

     \[
        \begin{tikzcd}
            X \arrow[r, "", hook] \arrow[dr, "f"]& G \arrow[d, "\tilde{f}", dashed]\\
                                                & A \rtimes G
        \end{tikzcd}
    \]
    where $f(x) = (\phi(x-1),x)$.
    Since  $G$ is a free group on  $X$, there exists a  $\tilde{f}$.
    Note, $\tilde{f}(g) =  (d(g), g)$, because generators are mapped to themselves in the second component.
    We have that $\tilde{f}$ is morphism of groups, so $d:G \to M$ is a derivation!

    Let $\psi \in  \Hom_{\Z G}(IG, A)$ corresponding to $d$.
    So  $\psi(x-1) = d(x)$ per definition.
    We have that  $d(x) = \phi(x-1)$, so  $\psi(x-1) = \phi(x-1)$.
    This shows that  $\tilde{\phi} = \psi$.
\end{proof}

\begin{corollary}[6.4.9]
    Let $F$ be a free group and let  $M$ by any  $F$-module.
    Then  
    \[
        H_i(F, M) =H^{i}(F, M) =0 \qquad i \ge 2
    .\] 
\end{corollary}
\begin{proof}
    \[
    \cdots \to  0 \to  0\to  0 \to  IF \xrightarrow{i}  \Z F \xrightarrow{\epsilon}  \Z \to  0
    .\] 
    This is a free $\Z F$-resolution of $\Z$ by the previous result.
\end{proof}
